\begin{enumerate}
    \item \label{CI} \textit{CI} (ang. \textit{Continuous integration}) - technika stosowana w trakcie rozwoju oprogramowania polegająca na włączaniu bieżących zmian w kodzie i każdorazowej weryfikacji zmian poprzez zbudowanie projektu.
    \item \label{CD} \textit{CD} (ang. \textit{Continous Delivery}) - proces automatycznego wdrażania aplikacji, gdzie testowany oraz budowany jest kod aplikacji. Na końcu publikujemy aplikację dla użytkowników końcowych.
    \item \label{IDE} \textit{IDE} (ang. \textit{Integrated Development Enviroment}) - jest to oprogramowanie, które umożliwia budowanie aplikacji wraz z zestawem narzędzi deweloperskich. W składzie IDE znajdują się takie elementy jak: edytor kodu, narzędzia do budowania kodu oraz debugger.
    \item \label{CSS} \textit{CSS} (ang. \textit{Cascading Style Sheets}) - jest to język programowania służący do opisu formy prezentacji strony www. W języku tym znajdują się reguły, które pozwalają zmienić danemu elementowi jego wygląd.
    \item \label{NoSQL} \textit{NoSQL} (ang. \textit{Non SQL}) - jest to rodzaj bazy danych, która nie jest oparta na języku SQL. W tym rodzaju bazach danych nie ma tabel, a dane są przechowywane w postaci klucz-wartość lub dokumentów. Dokumenty te posiadają różne typy danych, które mogą być zagnieżdżone.
    \item \label{HTML} \textit{HTML} (ang. \textit{HyperText Markup Language}) - jest to język znaczników, który służy do dokumentów hipertekstowych. Język ten jest używany do tworzenia stron internetowych.
    \item \label{JSON} \textit{JSON} (ang. \textit{JavaScript Object Notation}) - jest to format danych, który jest używany do przesyłania danych między aplikacjami. Ta notacja została stworzona na potrzeby języka JavaScript. Jest to format klucz-wartość, co pozwala na łatwiejsze poruszanie się po danych.
    \item \label{API} \textit{API} (ang. \textit{Application Programming Interface}) - jest to interfejs programowania, który umożliwia komunikację między różnymi aplikacjami. API jest często używane do komunikacji między aplikacjami internetowymi.
    \item \label{JWT} \textit{JWT} (ang. \textit{JSON Web Token}) - standard wymiany informacjami pomiędzy stronami za pomocą dokumentów \textit{JSON} \ref{JSON}. Jest to bezpieczny sposób na przesyłanie danych między stronami.
\end{enumerate}
