\begin{enumerate}
    \item \label{CI} \textit{CI} (ang. \textit{Continuous integration}) - technika stosowana w trakcie rozwoju oprogramowania polegająca na włączaniu bieżących zmian w kodzie i każdorazowej weryfikacji zmian poprzez zbudowanie projektu.
    \item \label{CD} \textit{CD} (ang. \textit{Continous Delivery}) - proces automatycznego wdrażania aplikacji, gdzie testowany oraz budowany jest kod aplikacji. Na końcu publikujemy aplikację dla użytkowników końcowych.
    \item \label{IDE} \textit{IDE} (ang. \textit{Integrated Development Enviroment}) - jest to oprogramowanie, które umożliwia budowanie aplikacji wraz z zestawem narzędzi deweloperskich. W skład IDE znajdują się takie elementy jak: edytor codu, narzędzia do budowania kodu oraz debugger.
    \item \label{CSS} \textit{CSS} (ang. \textit{Cascading Style Sheets}) - jest to język programowania, służący do opisu formy prezentacji strony www. W języku tym znajdują się reguły, które pozwalają zmienić danemu elementowi jego wygląd.
\end{enumerate}
