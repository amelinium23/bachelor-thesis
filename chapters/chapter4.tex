\section{Metoda testowania}
W tej sekcji przedstawione są metoda testowania, które zostały wykorzystane w projekcie. W tym rozdziale także zostały wymienione miejsca implementacji, które zostały przetestowane za pomocą tych metod. 

\subsection{Testy manualne}
Testy manualne jest to rodzaj testów, gdzie do ich wykonania potrzebny jest człowiek. Dzięki takiemu podejściu tester sprawdza daną funkcjonalność, pozwala to także na szybsze zgłoszenie błędów. Z danego testu są wygenerowane raporty, które są przesyłane do programisty, który ma za zadanie naprawić błędy. Dodatkowo w raporcie zostały załączone zrzuty ekranu, które pokazują dokładnie, gdzie wystąpił błąd. Atutem takiego rozwiązania jest także brak dodatkowej konfiguracji narzędzi potrzebnych do wykonania testów.

Podejście sprawia kilka problemów, najważniejszym z nich jest zdecydowanie większa ilość czasu, które jest potrzebne do wykonania testów. Dodatkowo, jeśli tester nie jest doświadczony, może popełnić błędy, które mogą być trudne do zauważenia. Ta własność powoduje także mniejszą dokładność. Wadą tego rozwiązania jest także brak możliwości uruchomienia dużej ilości testów jednocześnie, ponieważ jesteśmy ograniczeni zasobami ludzkimi. 

\section{Testowane miejsca implementacji}
W tej sekcji są przedstawiono przetestowanie miejsca implementacji. Wymienione są najważniejsze funkcjonalności, które zostały napisane w rozdziale \ref{ideal_solution}. Każda z osób miała za zadanie przetestować daną funkcjonalność.

W tej sekcji znalazły się następujące funkcjonalności:
\begin{enumerate}
  \item Logowanie i rejestracja użytkownika
  \item Dodawanie, edycja i usuwanie postów
  \item Edycja profilu użytkownika
  \item Weryfikacja maila użytkownika
  \item Wyświetlanie listy postów
  \item Sortowanie i wyświetlanie telefonów
  \item Sortowanie i wyświetlanie producentów
  \item Porównanie trzech telefonów
\end{enumerate}

\subsection{Logowanie i rejestracja użytkownika}
Aby móc zalogować się na stronę, należy posiadać przygotowany adres email. Następnie przechodzimy do ekranu logowania, za pomocą menu w prawym górnym rogu. Na rysunku \ref{login_menu} przedstawiono przycisk umożliwiający przejście do ekranu logowania.
\begin{figure}[H]
  \centering
  \includegraphics[scale=0.47]{tests/login-register/Login-button.png}
  \caption{Przycisk odpowiedzialny za przejście do ekranu logowania.}
  \label{login_menu}
\end{figure}
Następnie jesteśmy przenoszeni na ekran, na którym możemy się zalogować. Strona logowania została podzielona na dwie sekcje. Jedna z nich jest odpowiedzialna za logowanie. W tej sekcji znajduje się formularz gdzie należy wpisać adres email oraz hasło. Dodatkowo umieszczono także sekcję rejestracji, gdzie można wpisać email oraz hasło dla nowego użytkownika. Na rysunku \ref{login_page} zostały przedstawione obie sekcje.
\begin{figure}[H]
  \centering
  \includegraphics[scale=0.47]{tests/login-register/login-page.png}
  \caption{Strona logowania oraz rejestracji}
  \label{login_page}
\end{figure}
Aplikacja informuje użytkownika o wprowadzeniu nieprawidłowych danych. Aby poprawnie się zalogować, należy wpisać hasło, które ma przynajmniej 8 znaków, zawierać małe i duże litery, cyfry oraz znak specjalny. Aby móc się zarejestrować, należy wpisać adres email, który nie jest już zarejestrowany w bazie danych oraz hasło, które powinno zawierać przynajmniej 8 znaków, małe i duże litery, cyfry oraz znak specjalny. Po poprawnym zalogowaniu użytkownik jest przenoszony na stronę główną. Na rysunku \ref{login_success} przedstawiono przykład poprawnego logowania. 
\begin{figure}[H]
  \centering
  \includegraphics[scale=0.47]{tests/login-register/login-sucessful.png}
  \caption{Przykład poprawnego logowania}
  \label{login_success}
\end{figure}
Dodatkowo jeżeli użytkownik posiada zdjęcie profilowe jest ono wyświetlane w prawym górnym rogu strony. Do zdjęcia jest również dodany przycisk, który pozwala na przejście do profilu użytkownika.

W przypadku rejestracji jeżeli użytkownik wpiszę poprawne dane zostaje przeniesiony na stronę rejestracji. Ta strona zawiera także dodatkowe informacje, które użytkownik może wprowadzić. Są to informację na temat nazwy użytkownika oraz zdjęcia profilowego. Na rysunku \ref{register_page} przedstawiono stronę rejestracji.
\begin{figure}[H]
  \centering
  \includegraphics[scale=0.47]{tests/login-register/register-form.png}
  \caption{Formularz rejestracji}
  \label{register_page}
\end{figure}
Po poprawnej rejestracji użytkownik jest przenoszony na stronę główną. Dodatkowo użytkownik jest informowany za pomocą powiadomienia o poprawnym zarejestrowaniu wraz z nową nazwą użytkownika. 

Na rysunku \ref{success_register} przedstawiono przykład poprawnej rejestracji.
\begin{figure}[H]
  \centering
  \includegraphics[scale=0.45]{tests/login-register/register-success.png}
  \caption{Przykład poprawnego zarejestrowania}
  \label{success_register}
\end{figure}
W przypadku napotkanych problemów podczas użytkownik jest informowanych o tych w błędach za pomocą powiadomień. Wszystkie powiadomienia są wyświetlane w prawym dolnym rogu. W zależności od rodzaju powiadomienia są one rozróżnione za pomocą kolorów. Na rysunku \ref{register_error} przedstawiono przykład błędnego wprowadzenia danych.
\begin{figure}[H]
  \centering
  \includegraphics[scale=0.7]{tests/login-register/register-error.png}
  \caption{Błąd wyświetlany w przypadku błędu rejestracji.}
  \label{register_error}
\end{figure}
Użytkownik jest informowany o tym błędach w formie komunikatów w kontrolach. Jeżeli wartość danej kontrolki jest niepoprawna, to zostanie ona podświetlona na czerwono. Powoduje to zablokowanie przycisku uniemożliwiając możliwość zalogowania się na stronę.

Na rysunku \ref{login_error} przedstawiono przykład błędnego wprowadzenia danych.
\begin{figure}[H]
  \centering
  \includegraphics[scale=0.47]{tests/login-register/register-not-valid.png}
  \caption{Przykład błędnego wprowadzenia danych}
  \label{login_error}
\end{figure}

\subsection{Dodawanie, edycja i usuwanie postów}
Aby móc dodać post użytkownik musi być zalogowany. Dodawanie postów odbywa się po kliknięciu przycisku dodaj post, użytkownik zostaje przekierowany na stronę dodawania posta. Po kliknięciu przycisku użytkownik jest przenoszony na stronę dodawania posta. Przycisk ten pozostaje niewidoczny, dopóki użytkownik nie będzie zalogowany. Na rysunku \ref{add_post} przedstawiono stronę forum skąd można przejść do podstrony dodania posta.
\begin{figure}[H]
  \centering
  \includegraphics[scale=0.47]{tests/add-edit-delete-posts/add-post.png}
  \caption{Strona forum skąd można przejść do dodawania posta}
  \label{add_post}
\end{figure}
Strona zawiera kilka rodzajów wpisów, które można stworzyć na stronie. W aplikacji występują trzy rodzaje postów: 
\begin{enumerate}
  \item \textit{question} - użytkownik może zadać pytanie dotyczące telefonów, a inni użytkownicy mogą na nie odpowiedzieć.
  \item \textit{discussion} - użytkownik może stworzyć dyskusję na temat telefonów, gdzie inni użytkownicy mogą się wypowiedzieć na temat dyskusji.
  \item \textit{listing} - użytkownik może wystawić telefon w celu sprzedaży bądź też wyceny danego telefonu. Inni użytkownicy mogą ocenić telefon oraz zaproponować cenę.
\end{enumerate}
Na rysunku \ref{add_question_page} przedstawiono stronę dodawania wpisu typu \textit{question}.
\begin{figure}[H]
  \centering
  \includegraphics[scale=0.47]{tests/add-edit-delete-posts/create-question.png}
  \caption{Strona dodawania posta - \textit{question}}
  \label{add_question_page}
\end{figure}
Na rysunku \ref{add_discussion_post} przedstawiono stronę dodawania wpisu typu \textit{discussion}.
\begin{figure}[H]
  \centering
  \includegraphics[scale=0.47]{tests/add-edit-delete-posts/create-discussion.png}
  \caption{Strona dodawania posta - \textit{discussion}}
  \label{add_discussion_post}
\end{figure}
Na rysunku \ref{add_listing_post} przedstawiono stronę dodawania wpisu typu \textit{listing}.
\begin{figure}[H]
  \centering
  \includegraphics[scale=0.47]{tests/add-edit-delete-posts/create-listing.png}
  \caption{Strona dodawania posta - \textit{listing}}
  \label{add_listing_post}
\end{figure}
Na rysunkach \ref{add_discussion_post}, \ref{add_listing_post} oraz \ref{add_discussion_post} przedstawiono wyświetlanie błędów w przypadku niepoprawnego wprowadzenia danych. Można także zauważyć na tych rysunkach, że jeżeli występują błędy w formularzu, to przycisk dodawania posta jest zablokowany. Po wprowadzeniu poprawnych danych przycisk zostaje odblokowany. Na rysunku \ref{add_valid_post} przedstawiono poprawnie wypełniony formularz dodawania posta.
\begin{figure}[H]
  \centering
  \includegraphics[scale=0.47]{tests/add-edit-delete-posts/create-post-valid.png}
  \caption{Strona dodawania posta - poprawna walidacja pól}
  \label{add_valid_post}
\end{figure}
Po pomyślnym dodaniu posta użytkownik zostaje przeniesiony na stronę ze swoimi postami. Na rysunku \ref{add_post_success} przedstawiono pomyślne dodania posta.
\begin{figure}[H]
  \centering
  \includegraphics[scale=0.35]{tests/add-edit-delete-posts/successful-post-creation.png}
  \caption{Strona z postami użytkownika po dodaniu nowego posta}
  \label{add_post_success}
\end{figure}
Aby zmodyfikować post należy przejść na stronę z postami użytkownika i kliknąć przycisk \textit{Edit}. Aby usunąć post należy przejść na stronę z postami użytkownika i kliknąć przycisk \textit{Delete}. Na rysunku \ref{edit_post} przedstawiono stronę edycji posta. Aby zmodyfikować należy kliknąć przycisk \textit{Update post}. 
\begin{figure}[H]
  \centering
  \includegraphics[scale=0.46]{tests/add-edit-delete-posts/editting-post.png}
  \caption{Strona edycji posta}
  \label{edit_post}
\end{figure}
Jeżeli post zostanie poprawnie zmodyfikowany użytkownik zostaje przeniesiony do strony z postami użytkownika. Na rysunku \ref{edit_post_success} przedstawiono pomyślne zmodyfikowanie posta.
\begin{figure}[H]
  \centering
  \includegraphics[scale=0.36]{tests/add-edit-delete-posts/editted-post-success.png}
  \caption{Strona z postami użytkownika po pomyślnym edycji posta}
  \label{edit_post_success}
\end{figure}

\subsection{Edycja profilu użytkownika}
Aby zmodyfikować profil użytkownika należy przejść na stronę profilu użytkownika oraz być zalogowanym na stronie. Następnie przechodzimy za pomocą przycisku \textit{My profile}. Na rysunku \ref{profile_page} przedstawiono stronę profilu użytkownika.
\begin{figure}[H]
  \centering
  \includegraphics[scale=0.47]{tests/edit-profile/profile-page.png}
  \caption{Strona profilu użytkownika}
  \label{profile_page}
\end{figure}
Aby zmodyfikować profil należy kliknąć przycisk \textit{Edit profile}. Na rysunku \ref{edit_profile} przedstawiono stronę edycji profilu użytkownika. W formularzu można zmienić takie dane jak numer telefonu, zdjęcie oraz nazwa użytkownika. Aby zmodyfikować profil należy kliknąć przycisk \textit{Edit profile}.
\begin{figure}[H]
  \centering
  \includegraphics[scale=0.47]{tests/edit-profile/edit-profile-form.png}
  \caption{Strona edycji profilu użytkownika}
  \label{edit_profile}
\end{figure}
Dodatkowo na stronie profilu można zweryfikować adres email. Aby zweryfikować adres email należy kliknąć przycisk \textit{Verify email}. Następnie na wybrany adres email otrzymujemy maila. W mailu znajduje się link do zweryfikowania adresu email.  Na rysunku \ref{verify_email} przedstawiono stronę profilu użytkownika po zweryfikowaniu adresu email. Strona informuje o wysłaniu maila na podany adres email.
\begin{figure}[H]
  \centering
  \includegraphics[scale=0.45]{tests/edit-profile/send-verification-email.png}
  \caption{Strona profilu użytkownika po zweryfikowaniu adresu email}
  \label{verify_email}
\end{figure}
Na rysunku \ref{verify_email_message} przedstawiono email z linkiem do zweryfikowania adresu email. Ten mail został wygenerowany przez Firebase. 
\begin{figure}[H]
  \centering
  \includegraphics[scale=0.32]{tests/edit-profile/verification-mail.png}
  \caption{Email z linkiem do zweryfikowania adresu email}
  \label{verify_email_message}
\end{figure}
Po kliknięciu w link użytkownik zostaje przekierowany do strony weryfikacji stworzonej przez Firebase. Firebase pozwala na zmodyfikowanie treści maila. Na rysunku \ref{verify_email_page} przedstawiono stronę weryfikacji adresu email.
\begin{figure}[H]
  \centering
  \includegraphics[scale=0.33]{tests/edit-profile/mail-verification-successfull.png}
  \caption{Strona weryfikacji adresu email}
  \label{verify_email_page}
\end{figure}
Użytkownik także posiada edytowanie używanego przez siebie urządzenia. Aby zmodyfikować urządzenie należy przejść na stronę profilu użytkownika oraz być zalogowanym na stronie. Następnie po wciśnięciu przycisku \textit{Edit Phone} użytkownik wybierze swoje urządzenie poprzez rozwijane menu. W tym menu zostają wyświetlone dostępne na stronie urządzenia. Na rysunku \ref{edit_phone} przedstawiono stronę edycji urządzenia.
\begin{figure}[H]
  \centering
  \includegraphics[scale=0.47]{tests/edit-profile/phone-list.png}
  \caption{Strona edycji urządzenia}
  \label{edit_phone}
\end{figure}
Po pomyślnej zmianie urządzenia na stronie powiada się powiadomienie o pomyślnej zmianie urządzenia. Na rysunku \ref{edit_phone_success} przedstawiono stronę profilu użytkownika po pomyślnej zmianie urządzenia.
\begin{figure}[H]
  \centering
  \includegraphics[scale=0.44]{tests/edit-profile/updated-user-device.png}
  \caption{Strona profilu użytkownika po pomyślnej zmianie urządzenia}
  \label{edit_phone_success}
\end{figure}

\subsection{Sortowanie i wyświetlanie telefonów}
Aby wyświetlić listę telefonów należy przejść na podstronę o nazwie \textit{Phones}. Na tej stronie wyświetlane są wszystkie dostępne na stronie telefony. Telefony na stronie są grupowane ze względu na producenta. Na rysunku \ref{phones_page} przedstawiono stronę z listą telefonów.
\begin{figure}[H]
  \centering
  \includegraphics[scale=0.42]{tests/display-phones/phone-page.png}
  \caption{Strona z listą telefonów}
  \label{phones_page}
\end{figure}
Zastosowana technika \textit{lazy loading} pozwoliła na opóźnienie władowania zdjęć telefonów. Technika ta pozwala na opóźnienie ładowania statycznych plików jak zdjęcia. Obecnie przeglądarki obsługują przechowywanie zdjęć w pamięci podręcznej, co pozwala na ładowanie zdjęć z pamięci podręcznej. Pozwoliło to na szybsze ładowanie strony. Na serwer zostają przesyłane parametry jak: numer strony, liczba elementów na stronie oraz tryb sortowania. Użytkownik na stronie może zmienić tryb sortowania. Język Python udostępnia możliwość łatwą możliwość paginacji. Ta metoda nazywa się cięciem. W algorytmie \ref{slice_example} przedstawiono  został przedstawiony kod odpowiedzialny za paginację telefonów.
\begin{code}[H]
  \begin{minted}[
    frame=lines,
    framesep=2mm,
    baselinestretch=1.2,
    autogobble=true,
    breaklines,
    fontsize=\footnotesize
  ]{python3}
  @DEVICE.route("/")
  def get_all_devices_by_brand() -> Response:
      try:
          page_number = args.get("page_number")
          page_size = args.get("page_size")
          sort_mode = args.get("sort_mode", "ascending")
          devices = get_device_list_by_brands(db)
          sorted_devices = sort_devices(devices, sort_mode)
          result = {
              "data": sorted_devices,
              "total": len(devices),
              "totalPhones": count_phones(devices),
          }
          if page_size and page_number and sort_mode:
              start_index: int = (int(page_number) - 1) * int(page_size)
              end_index: int = start_index + int(page_size)
              result = {
                  "data": sorted_devices[start_index:end_index],
                  "total": len(devices),
                  "totalPhones": count_phones(devices),
              }
          return jsonify(result)
      except Exception as e:
          return Response(str(e), status=500)    
  \end{minted}
  \centering
  \caption{Endpoint znajdujący się w aplikacji - pozwala na paginację wykonaną metodą cięcia}
  \label{slice_example}
\end{code}
Aby zmienić tryb sortowania należy wybrać odpowiednią opcję z rozwijanego menu. Na rysunku \ref{reverse_sorting} przedstawiono zmianę trybu sortowania na \textit{Descending}.
\begin{figure}[H]
  \centering
  \includegraphics[scale=0.42]{tests/display-phones/phone-sorting-reverse.png}
  \caption{Strona z listą telefonów po zmianie trybu sortowania}
  \label{reverse_sorting}
\end{figure}

\subsection{Porównanie trzech telefonów}
Aby porównać telefony należy przejść na podstronę o nazwie \textit{Compare}. Użytkownik wybiera za pomocą rozwijanych menu trzy urządzenia. Urządzenia pobierane są z serwera, a następnie wyświetlane są na stronie. Każde wybranie urządzenia powoduje władowanie informacji o urządzeniu. Na rysunku \ref{compare_page} przedstawiono stronę porównania przed wybraniem urządzeń do porównania.
\begin{figure}[H]
  \centering
  \includegraphics[scale=0.42]{tests/compare-devices/compare-page.png}
  \caption{Strona z listą telefonów po zmianie trybu sortowania}
  \label{compare_page}
\end{figure}

Aby pobrać informacje o telefonach wykorzystuje statyczną metodę występującą w JavaScript. Metoda ta nazywa się \textit{Promise.all}. Metoda ta pozwala na wykonanie kilku zapytań asynchronicznych jednocześnie. Pozwala to na szybsze pobieranie informacji o urządzeniach. Te dane są zwracane w formie tablicy. W algorytmie \ref{getting_devices} przedstawiono zastosowanie metody \textit{Promise.all}.

\begin{code}[H]
  \begin{minted}[
    frame=lines,
    framesep=2mm,
    baselinestretch=1.2,
    autogobble=true,
    breaklines,
    fontsize=\footnotesize,
    ]{Typescript}
    const getComparison = async (deviceKeys: string[]) => {
      const comparisons: { [key: string]: DeviceDetails } = {}
      const promises = deviceKeys.map(async (key) => {
        return await getDetails(key)
      })
      const data = await Promise.all(promises)
      data.forEach((device: DeviceDetails) => {
        comparisons[device.key] = device
      })
      return comparisons
    }
  \end{minted}
  \caption{Endpoint znajdujący się w aplikacji - pozwala na paginację wykonaną metodą cięcia}
  \label{getting_devices}
\end{code}
Poprzez wykorzystanie języka TypeScript można było zdefiniować dokładnie jakie typy danych możemy się spodziewać w odpowiedzi. W tym przypadku oczekiwany jest obiekt, który zawiera klucz oraz wartość będącą obiektem typu \textit{DeviceDetails}.

Po wybraniu urządzenia użytkownik może porównać wybrane urządzenia. Informacje dotyczące urządzeń są przedstawione w formie tabeli. Na rysunku \ref{compare_three_devices} przedstawiono stronę porównania po wybraniu urządzeń.

\begin{figure}[H]
  \centering
  \includegraphics[scale=0.36]{tests/compare-devices/compare-three-devices.png}
  \caption{Wykonanie porównania trzech urządzeń}
  \label{compare_three_devices}
\end{figure}
