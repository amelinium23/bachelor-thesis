\section{Problematyka}
Codziennie na portalach aukcyjnych pojawiają się setki nowych urządzeń, co skutkuje wystawianiem starszej elektroniki na sprzedaż. Każdy posiadacz elektroniki użytkowej inaczej używa, co powoduje kłopot w kupnie takiego sprzętu. Nie mamy pojęcia, czego możemy się spodziewać po sprzęcie używanym.

Dodatkowo mamy dużo miejsc, gdzie możemy uzyskać informacji na temat danej elektroniki. Dziennikarze technologiczni codziennie produkują recenzję czy testy danego sprzętu. Recenzenci starają się obiektywnie oceniać dany sprzęt elektroniczny, jednakże możemy teraz zauważyć dużą dezinformacje. Powodem tego są płatne recenzję przez producentów elektroniki. Z tego względu dane recenzje stają się mało wiarygodne oraz warte uwagi.

Każdy sprzęt używany miał kiedyś swojego właściciela, który użytkował na swój sposób. Ten sposób użytkowania telefonu wpływa bezpośrednio na komponenty telefonu. Aktualnie w technologii produkowania telefonów komórkowych można zauważyć trend, gdzie posiadają one składane ekrany. Każdy z tych ekranów posiada daną liczbę złożeń zanim przestanie się składać. Warto tutaj powrócić do recenzentów, otóż nie mogą oni sprawdzić liczby złożeń. Powodów jest kilka: ograniczony czas wypożyczenia sprzętu, sprzęt także trafia od recenzenta do recenzenta, co powoduje, że te telefony są bardzo narażone na uszkodzenia. Tutaj przychodzą producenci, którzy mogą pomiędzy recenzentami zmienić telefon, takie zachowanie powoduje, że konsument takie sprzętu nie wie jaka jest rzeczywista liczba możliwych złożeń.

Jednym z rozwiązaniem na ten problem są fora internetowe lub grupy na portalach społecznościowych. Użytkownicy takich forów czy grup chętniej dzielą się problemami danego sprzętu. Natomiast żadna redakcja, nie daje użytkownikom możliwości wypowiedzenia się na temat używania danego urządzenia. Nie występuje także możliwość wyceny danego używanego sprzętu, aby osoba sprzedająca dane urządzenia mogła usłyszeć proponowaną cenę według obiektywnego eksperta.

\section{Cel pracy}
Celem niniejszej pracy jest opracowanie i implementacja aplikacji webowej, w oparciu o Flask i React, z użyciem Firebase \cite*{firebase}, Firestore \cite*{firestore} oraz Google Cloud Storage \cite*{cloud_storage}. Aplikacja pozwala na przeglądanie danych technicznych telefonów komórkowych, dodatkowo umożliwia dodawania postów oraz komentarzy na forum. Posty użytkownika będą udostępnione do edycji bądź też usuwania. Każdy użytkownik posiada możliwość wystawienia oferty danego urządzenia. W przypadku wystawienia ogłoszenia, aplikacja udostępnia proponowane ceny dla urządzenia, którego dotyczy ogłoszenie. Aplikacja pozwala także, na porównywanie wybranych przez użytkownika telefonów komórkowych.

\section{Zakres pracy}
Zakres pracy obejmuje zbudowanie aplikacji w oparciu o język Python \cite{python} i Typescript\cite*{TypeScript}. Część backendowa została zbudowana przy użyciu biblioteki Flask \cite{flask}, natomiast część frontendowa została zbudowana przy użyciu biblioteki React. W aplikacji została wykorzystana baza danych Firestore wraz z Google Cloud Store jako rozwiązanie do przechowywania obrazów. 
