\section{Cechy rozwiązania autora}
\label{ideal_solution}
Codziennie na rynku pojawiają się nowe telefony komórkowe. Wybór odpowiedniego sprzętu dla kupującego to skomplikowane zadanie. Osoba kupująca telefon musi sprecyzować, jakie są dokładnie jej potrzeby. Czy telefon, który chce zakupić potrzebuje dobrego aparatu, jakiego czasu pracy oczekuje od swojego przyszłego urządzenia, ile pieniędzy może przeznaczyć na nowe urządzenie. Na wyżej wymienione pytania osoba kupująca musi odpowiedzieć, a następnie dokonuje decyzji zakupowej.

Wiele osób kupuje używane telefony komórkowe, ponieważ są one znacznie tańsze od nowych. Powstaje problem, gdzie można znaleźć informacje o używanych jak i nowych telefonach komórkowych. Przy kupnie używanego telefonu brakuje informacji od użytkowników o tym, jak długo telefon działał.  

Zatem cechy jakie powinna posiadać. Strona, która spełniałby oczekiwania osoby zainteresowanej kupnem telefonu komórkowego:
\begin{enumerate}
    \item aktualne dane z nadchodzącymi premierami telefonami;
    \item recenzje o telefonach komórkowych;
    \item artykuły związane z telefonami komórkowymi;
    \item forum wraz z opiniami użytkowników;
    \item ocena użytkowników danego telefonu komórkowego;
    \item opinie użytkowników na temat używanych telefonów komórkowych;
    \item możliwość porównania telefonów komórkowych;
\end{enumerate}

\section{Dostępne rozwiązania}
W tej sekcji znajduje się spis dostępnych rozwiązań. Rozwiązania te zostaną dodatkowo porównane do rozwiązania autora, dzięki czemu możemy zobaczyć, czy dana strona sprosta wymaganiom użytkowników.

\subsection{GSM Arena}
\href{https://www.gsmarena.com/}{GSM Arena} \cite{gsm_arena} jest to anglojęzyczny portal dotyczący telefonów komórkowych. W tym serwisie możemy przeglądać przeróżne urządzenia. Na stronie znajdują się artykuły z zapowiedziami nowych urządzeń. Serwis ten posiada również pobieranie cen z serwisów aukcyjnych takich jak: Amazon, AliExpress czy EBay.

Rysunek \ref*{GSM_Arena_1} przedstawia dane techniczne telefonu OnePlus 10 Pro wraz z jego aktualnymi cenami.
\begin{figure}[H]
    \centering
    \includegraphics[scale=0.38]{img/GSM Arena/DetailsPageWithPrices.png}
    \caption{GSM Arena - Dane techniczne dla telefonu komórkowego: OnePlus 10 Pro}
    \label{GSM_Arena_1}
\end{figure}
Atutem tego serwisu są zdjęcia z danego telefonu, gdzie każdy użytkownik może sam ocenić zdjęcia w pełnej rozdzielczości. Jest to bardzo ważne dla amatorów fotografii. Rysunek \ref*{GSM_Arena_2} przedstawia zdjęcia w dobrych warunkach oświetleniowych z OnePlus 10 Pro. 
\begin{figure}[H]
    \centering
    \includegraphics[scale=0.76]{img/GSM Arena/ReviewWithPhotos.png}
    \caption{GSM Arena - zdjęcia w dobrych warunkach oświetleniowych z OnePlus 10 Pro}
    \label{GSM_Arena_2}
\end{figure}
GSM Arena posiada zdjęcia w różnych warunkach oświetlenia, które są bardzo pomocne przy wyborze telefonu. 

Rysunek \ref*{GSM_Arena_3} przedstawia zdjęcia z OnePlus 10 Pro w słabych warunkach oświetleniowych.
\begin{figure}[H]
    \centering
    \includegraphics[scale=0.7]{img/GSM Arena/GSMArena_picture_sample_lowlight.png}
    \caption{GSM Arena - zdjęcia w słabych warunkach oświetleniowych z OnePlus 10 Pro}
    \label{GSM_Arena_3}
\end{figure}
Portal ten nie posiada natomiast forum, gdzie użytkownicy mogliby się wypowiadać na dany temat. GSM Arena pozwala na wystawienie opinii tekstowej. Opinie te są pisane w języku angielskim. Dodatkowo użytkownik w swojej opinii może cytować także inne opinie. Efektem tego działania jest zmniejszenie powtarzalnych opinii. Brakuje możliwości oceny telefonu według zadanych kryteriów jak czas pracy na baterii czy budowa urządzenia.

Strona posiada tylko możliwość zapisania telefonu do listy ulubionych. Funkcja ta jest przydatna, dzięki czemu użytkownik może zapamiętać i później sprawdzić dane techniczne telefonu.

Rysunek \ref*{GSM_Arena_4} przedstawia najważniejsze dane techniczne wraz z możliwością dodania telefonu do listy ulubionych.

\begin{figure}[H]
    \centering
    \includegraphics[scale=0.77]{img/GSM Arena/GSMArena_become_a_fan.png}
    \caption{GSM Arena - Dodanie telefonu do listy ulubionych}
    \label{GSM_Arena_4}
\end{figure}
Na stronie są wyświetlane komentarze, jednakże ich jakość jest nietrafiona i o wątpliwej jakości. Nie wypowiadają się użytkownicy, którzy potwierdzili posiadanie danego urządzenia. Dla osoby kupującej używany telefon komórkowy, nie jest to korzystne, ponieważ nie może dowiedzieć się o mankamentach danego telefonu bądź też degradacji baterii.

Na rysunku \ref*{GSM_Arena_5} przedstawiono opinie użytkowników o OnePlus 10 Pro.
\begin{figure}[H]
    \centering
    \includegraphics[scale=0.53]{img/GSM Arena/Comments.png}
    \caption{GSM Arena - Opinie o OnePlus 10 Pro}
    \label{GSM_Arena_5}
\end{figure}

W tabeli \ref*{comparison_gsm_arena} znajduje się porównanie rozwiązań, które zostały zauważone na stronie GSM Arena do rozwiązania autora. Widzimy jednakże, że brakuje cech określonych w wymaganiach rozwiązania autora.
\begin{table}[H]
    \centering
    \begin{tabular}{|l|c|}
        \hline
        \multicolumn{1}{|c|}{Cecha rozwiązania} & \multicolumn{1}{c|}{GSM Arena} \\ \hline
        Aktualne dane z nadchodzącymi premierami telefonami & $x$ \\ \hline
        Recenzje telefonów komórkowych & $x$ \\ \hline
        Forum wraz z opiniami użytkowników & - \\ \hline
        Ocena użytkowników danego telefonu komórkowych & $x$ \\ \hline
        Opinie użytkowników na temat używanych telefonów komórkowych & - \\ \hline
        Możliwość porównania telefonów komórkowych & $x$ \\ \hline
    \end{tabular}
    \caption{Podsumowanie cech rozwiązania GSM Areny}
    \label{comparison_gsm_arena}
\end{table}
W tabeli \ref*{comparison_gsm_arena} zostały przyjęte oznaczenia: \textit{x} oznacza, że dane rozwiązanie ma pożądaną cechę, \textit{-} (minus) oznacza brak danej cechy w przedstawionym rozwiązaniu.

Podsumowując, GSM Arena jest to portal, który najbardziej przypomina rozwiązanie autora. Do rozwiązania brakuje mu cech, który umożliwiłby kupno urządzeń używanych na stronie. Dużym atutem portalu jest posiadanie informacji z zewnętrznych źródeł, co pozwala na sprawdzenie aktualnych cen telefonów.

\subsection{mgsm.pl}
\href{https://www.mgsm.pl/pl/}{mgsm.pl} \cite{mgsm} jest to polskojęzyczny portal dotyczący telefonów komórkowych. Możemy w nim zobaczyć przeróżne testy telefonów komórkowych, dodatkowo są opakowane w formę tekstową oraz video. Strona ta posiada stronę z aktualnościami, gdzie umieszczane są aktualne pogłoski na temat nowych urządzeń czy też wiadomości dotyczące nowych premier telefonów komórkowych.
Na rysunku \ref*{mgsm_1} przedstawiono główną stronę portalu.
\begin{figure}[H]
    \centering
    \includegraphics[scale=0.4]{img/mgsm/mgsm.png}
    \caption{mgsm.pl - Strona główna}
    \label{mgsm_1}
\end{figure}
Na stronie znajdują się przeróżne rankingi dotyczące popularności danego telefonu. Strona także udostępnia informacje o tym, ile wizyt ma dany telefon komórkowych, co jest wygodne dla użytkownika, bo w prosty sposób pokazuje, kiedy odbędzie się premiera nowego telefonu.

Na rysunku \ref*{mgsm_2} przedstawiono ranking telefonów komórkowych wraz z liczbą wizyt.
\begin{figure}[H]
    \centering
    \includegraphics[scale=0.4]{img/mgsm/rankingsMgsm.png}
    \caption{mgsm.pl - Rankingi wraz z łączną liczbą wizyt stronnie dziennie.}
    \label{mgsm_2}
\end{figure}
W dokładny i przejrzysty sposób są ukazane dane techniczne urządzeń. Co więcej w każdej ze specyfikacji telefonów są ukazane zdjęcia z danego urządzenia. Polskojęzyczne portale, nie dysponują połączeniem z zagranicznymi serwisami aukcyjnymi jak Amazon czy Ebay. Mamy tutaj dostęp tylko i wyłącznie do sklepu Media expert, więc możliwości tego portalu są ograniczone do jednego sklepu.

Brakuje także dostępu do polskich serwisów aukcyjnych takich jak: Allegro czy OLX, gdzie użytkownicy mogliby wystawiać ogłoszenia dotyczące sprzedaży ich użytkowników.

Na rysunku \ref*{mgsm_3} przedstawiono dane techniczne telefonu Samsung Galaxy S20FE.
\begin{figure}[H]
    \centering
    \includegraphics[scale=0.48]{img/mgsm/DetailsMgsm.png}
    \caption{mgsm.pl - Dane techniczne urządzenia: Samsung Galaxy S20FE}
    \label{mgsm_3}
\end{figure}
Możemy także zauważyć dołączoną do urządzenia recenzję wideo, która znajduje się na platformie Youtube. Dzięki tej funkcji użytkownik ma możliwość wysłuchania opinii na temat danego telefonu komórkowego. Można także w danych zauważyć reklamy, które nakłaniają nas do kupienia akcesoriów do wybranego urządzenia.

Kolejną zaletą tego portalu są bardziej obszerne opinie użytkowników na temat danego urządzenia. Są one oznaczane jako: pozytywne, negatywne lub neutralne. Dodatkowo użytkownicy opisują dokładnie zalety i wady danego urządzenia, które są oparte na użytkownikach, którzy mieli styczność z danym urządzeniem.  

Na rysunku \ref*{mgsm_4} przedstawiono przykładową stronę z opiniami użytkowników.
\begin{figure}[H]
    \centering
    \includegraphics[scale=0.45]{img/mgsm/mgsmComments.png}
    \caption{mgsm.pl - Opinia dotycząca urządzenia Samsung Galaxy S20FE}
    \label{mgsm_4}
\end{figure}
Inną zaletą tego portalu jest zawarty słowniczek dla osób, które nie mają do czynienia na co dzień z technologią. 
\begin{figure}[H]
    \centering
    \includegraphics[scale=0.47]{img/mgsm/dictonary.png}
    \caption{mgsm - Słowniczek przydatnych zwrotów }
    \label{mgsm_5}
\end{figure}
W tabeli \ref*{comparison_mgsm} znajduje się porównanie rozwiązań, które zostały zauważone na stronie mgsm.pl do rozwiązania autora. Widzimy jednakże, że brakuje cech określonych w wymaganiach rozwiązania autora.
\begin{table}[H]
    \centering
    \begin{tabular}{|l|c|}
        \hline
        \multicolumn{1}{|c|}{Cecha rozwiązania} & \multicolumn{1}{c|}{mgsm.pl} \\ \hline
        Aktualne dane z nadchodzącymi premierami telefonami & $x$ \\ \hline
        Recenzje telefonów komórkowych & $x$ \\ \hline
        Forum wraz z opiniami użytkowników & - \\ \hline
        Ocena użytkowników danego telefonu & $x$ \\ \hline
        Opinie użytkowników na temat używanych telefonów komórkowych & - \\ \hline
        Możliwość porównania telefonów komórkowych & $x$ \\ \hline
    \end{tabular}
    \caption{Podsumowanie cech rozwiązania autora i msgm.pl}
    \label{comparison_mgsm}
\end{table}
W tabeli \ref*{comparison_mgsm} przyjęto oznaczenia: \textit{x} oznacza, że dane rozwiązanie ma pożądaną cechę, \textit{-} (minus) oznacza brak danej cechy w przedstawionym rozwiązaniu.

Podsumowując mgsm.pl jest to strona bardzo podobna do GSM Areny, które zawiera dodatkowe możliwości. Atutem tej strony jest słowniczek wyjaśniający poszczególne pojęcia dotyczące telefonów komórkowych. Wadą tego serwisu jest integracja tylko z jednym sklepem. Użytkownik, może uznać portal za mało obiektywny. Co więcej, brakuje także integracji z polskimi serwisami aukcyjnymi. Pozwoliłoby to na rozszerzenie możliwości portalu, jednocześnie zwiększając jego popularność.

Jednakże temu rozwiązaniu brakuje możliwości, które posiada rozwiązanie autora. Brakuje także opinii użytkowników na temat używanych telefonów komórkowych. 

\subsection{Kimovil}
\href{https://www.kimovil.com/pl/}{Kimovil} \cite{kimovil} jest to wielojęzyczna przeglądarka sprzętu elektronicznego, na stronie znajduje się także inny sprzęt elektroniczny taki jak telewizory czy tablety. Dodatkowym atutem strony są zniżki czy promocje na zakup sprzętu elektronicznego. Strona udostępnia także newsletter, aby móc obserwować różne promocje na dany sprzęt elektroniczny.

Na rysunku \ref*{kimovil_1} przedstawiono główną stronę portalu.
\begin{figure}[H]
    \centering
    \includegraphics[scale=0.36]{img/Kimovil/kimovil.png}
    \caption{Kimovil - Strona główna}
    \label{kimovil_1}
\end{figure}
Na stronie są zamieszczone również rankingi, gdzie użytkownicy mogą oznaczać urządzenia jako urządzenie, które pożądają. Dodatkowym elementem jest ranking urządzeń, które użytkownicy posiadają lub posiadali w przeszłości.

Na rysunku \ref*{kimovil_2} przedstawiono rankingi znajdujące się na stronie.
\begin{figure}[H]
    \centering
    \includegraphics[scale=0.448]{img/Kimovil/rankingsKimovil.png}
    \caption{Kimovil - rankingi}
    \label{kimovil_2}
\end{figure}
Na stronie znajdują się dane techniczne danego urządzenia, które w przejrzysty sposób pokazują stosunek ceny do jakości według serwisu Kimovil. Co więcej, można zauważyć, że przy danych technicznych mamy wymienione sklepy wraz z najkorzystniejszą ofertą dla danego telefonu. 

Przy danych technicznych mamy także dostępne podsumowanie, gdzie ukazana jest np. nazwa matrycy oraz procesor, które zostały użyte w danym telefonie. Ważną informacją dla użytkownika jest także informacja o używanych pasmach sieci telekomunikacyjnych z rozróżnieniem na sieci. Jest to szczególnie przydatne w Polsce, ze względu na obsługiwanie danych pasm przez operatorów sieci komórkowych.

Na rysunku \ref*{kimovil_3} przedstawiono dane techniczne wraz z podsumowaniem telefonu Xiaomi Redmi K50 Gaming Edition.

\begin{figure}[H]
    \centering
    \includegraphics[scale=0.45]{img/Kimovil/kimovilDetails.png}
    \caption{Kimovil - Dane techniczne oraz podsumowanie dla urządzenia: Xiaomi Redmi K50 Gaming Edition}
    \label{kimovil_3}
\end{figure}

Dla każdego urządzenia udostępniane są przez użytkowników opinie. Każda opinia czy też komentarz są udzielone wraz z oceną danego komponentu. Znajdują się tam dodatkowe informacje dotyczące czasu pracy na baterii lub też, informacja ile dany użytkownik posiada dany sprzęt. Dodatkowo każdy telefon oraz każdy jego komponent jest oceniany w skali od 1-10.

Na rysunku \ref*{kimovil_4} przedstawiono przykładowe opinie na temat telefonu Xiaomi Redmi K50 Gaming Edition.
\begin{figure}[H]
    \centering
    \includegraphics[scale=0.5]{img/Kimovil/opinieKimovil.png}
    \caption{Kimovil - Opinie dla urządzenia: Xiaomi Redmi K50 Gaming Edition}
    \label{kimovil_4}
\end{figure}
W tabeli \ref*{comparison_kimovil} znajduje się porównanie rozwiązań, które zostały zauważone na stronie Kimovil do rozwiązania autora. Widzimy jednakże, że brakuje cech określonych w wymaganiach rozwiązania autora.
\begin{table}[H]
    \centering
    \begin{tabular}{|l|c|}
        \hline
        \multicolumn{1}{|c|}{Cecha rozwiązania} & \multicolumn{1}{c|}{Kimovil} \\ \hline
        Aktualne dane z nadchodzącymi premierami telefonami & $x$  \\ \hline
        Recenzje telefonów komórkowych & - \\ \hline
        Forum wraz z opiniami użytkowników & - \\ \hline
        Ocena użytkowników danego telefonu komórkowego & $x$ \\ \hline
        Opinie użytkowników na temat używanych telefonów komórkowych & $x$ \\ \hline
        Możliwość porównania telefonów komórkowych & $x$ \\ \hline
    \end{tabular}
    \caption{Podsumowanie cech rozwiązania Kimovil}
    \label{comparison_kimovil}
\end{table}
W tabeli \ref*{comparison_kimovil} przyjęto oznaczenia: \textit{x} oznacza, że dane rozwiązanie ma pożądaną cechę, \textit{-} (minus) oznacza brak danej cechy w przedstawionym rozwiązaniu.

Podsumowując, Kimovil jest to rozwiązanie, które przypomina rozwiązanie autora jednakże brakuje mu kilku cech, które ma rozwiązanie autora. Na stronie Kimovil brakuje recenzji telefonów, które pomogłyby w wybraniu danego urządzenia. Dodatkową wadą jest brak forum, gdzie użytkownicy mogą wypowiedzieć się na temat urządzeń. Atutem jest posiadanie wielu języków wraz z odpowiadającymi dla kraju walutami.

\subsection{Versus}
\href{https://versus.com/}{Versus} \cite{versus} jest wielojęzyczną porównywarką. Atutem tej porównywarki, tak samo jak w przypadku Kimovil, jest możliwość porównywania innego sprzętu elektronicznego. Na stronie również można na podstawie opinii użytkowników porównać także miasta czy uczelnie.

Na tym portalu znajduje się duża baza telefonów, jednakże dane są uzupełniane przez użytkowników, co powoduje, że dane mogą się różnić od tych, które dostarcza producent.

Na rysunku \ref*{versus_1} przedstawiono rankingi znajdujące się na stronie głównej.
\begin{figure}[H]
    \centering
    \includegraphics[scale=0.45]{img/versus/versusRankingi.png}
    \caption{Versus - Rankingi}
    \label{versus_1}
\end{figure}
Na stronie ukazana jest specyfikacja danych urządzeń, w postaci podsumowania. Dane dotyczące specyfikacji pochodzą od użytkowników. Użytkownik po wejściu na stronę danego urządzenia jest witany podsumowaniem dotyczącym danego urządzenia. W tym podsumowaniu możemy zobaczyć najważniejsze dane telefonu tj.: informacje o aparacie, ekranie czy pamięci operacyjnej.
\newpage
Na rysunku \ref*{versus_2} przedstawiono przykładowe podsumowanie dla Samsunga Galaxy S22 Ultra.
\begin{figure}[h]
    \centering
    \includegraphics[scale=0.46]{img/versus/versusDetails.png}
    \caption{Versus - Podsumowanie danych dotyczące telefonu Samsung Galaxy S22 Ultra}
    \label{versus_2}
\end{figure}
Przy każdym urządzeniu umieszczone są różne opinie użytkowników. Opinie te dzielą się na poszczególne komponenty danego urządzenia. Każdy użytkownik, ocenia dany komponent w skali od 1-10. Co więcej, każda opinia ma osobisty komentarz użytkownika, w którym dany użytkownik wymienia wady i zalety oraz problemy, które napotkał z danym urządzeniem.

Na rysunku \ref*{versus_3} przedstawiono przykładową ocenę urządzenia z podziałem na kategorie.
\begin{figure}[H]
    \centering
    \includegraphics[scale=0.45]{img/versus/versusOpinie.png}
    \caption{Versus - Ocena poszczególnych kategorii dla urządzenia Samsung Galaxy S22 Ultra}
    \label{versus_3}
\end{figure}
Na rysunku \ref*{versus_4} przedstawiono przykładowy komentarz użytkownika dla Samsung Galaxy S22 Ultra.
\begin{figure}[H]
    \centering
    \includegraphics[scale=0.45]{img/versus/versusKomentarze.png}
    \caption{Versus - Opinia dotycząca urządzenia Samsung Galaxy S22 Ultra}
    \label{versus_4}
\end{figure}

W tabeli \ref{comparison_versus} znajduje się porównanie rozwiązania do rozwiązania autora. Widzimy jednakże, że brakuje cech, które definiowałyby rozwiązanie autora, które znajdują się w sekcji \ref{ideal_solution}.
\begin{table}[H]
    \centering
    \begin{tabular}{|l|c|}
        \hline
        \multicolumn{1}{|c|}{Cecha rozwiązania} & \multicolumn{1}{c|}{Versus} \\ \hline
        Aktualne dane, z nadchodzącymi premierami telefonami & $x$ \\ \hline
        Recenzje telefonów komórkowych & - \\ \hline
        Forum wraz z opiniami użytkowników & - \\ \hline
        Ocena użytkowników danego telefonu komórkowego & $x$ \\ \hline
        Opinie użytkowników na temat używanych telefonów komórkowych & - \\ \hline
        Możliwość porównania telefonów komórkowych & $x$ \\ \hline
    \end{tabular}
    \caption{Podsumowanie cech rozwiązania Versus}
    \label{comparison_versus}
\end{table}
W tabeli \ref{comparison_versus} przyjęto oznaczenia: \textit{x} oznacza, że dane rozwiązanie ma pożądaną cechę, \textit{-} (minus) oznacza brak danej cechy w przedstawionym rozwiązaniu.

Versus jest to rozwiązanie, które głównie ma spełniać funkcjonalność porównywarki. Brakuje mu kilku cech, które posiadałoby rozwiązanie zbudowane przez autora w ramach pracy inżynierskiej. Dużą wadą tego serwisu jest uzupełnianie przez użytkowników danych, które nie zawsze okazują się prawdziwe.
