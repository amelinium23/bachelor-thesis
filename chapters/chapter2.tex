\section{Cechy rozwiązanie autora}
\label{ideal_solution}
Codziennie na rynku pojawiają się nowe telefony komórkowe. Dla kupującego telefon powoduje to skomplikowane zadanie dla przyszłego kupca. Osoba kupująca telefon musi sprecyzować, jakie są dokładnie jego potrzeby. Czy telefon, który chcę zakupić potrzebuje dobrego aparatu? Jakiego czasu pracy oczekuję od swojego przyszłego urządzenia? Ile pieniędzy może przeznaczyć na nowe urządzenie? Na wyżej wymienione pytania osoba kupująca odpowiada, a następnie dokonuje decyzji zakupowej.

Wiele osób kupuje używane telefony komórkowe, ponieważ są one znacznie tańsze od nowych. Powstaję problem, gdzie można znaleźć informację o używanych jak i nowych telefonach komórkowych. Przy kupnie używanego telefonu brakuje informacji od użytkowników o tym, jak długo telefon działał.  

Zatem jakie cechy powinna posiadać taka strona, która spełniałby oczekiwanie osoby zainteresowanej kupnem telefonu komórkowego.
\begin{enumerate}
    \item Aktualne dane z nadchodzącymi premierami telefonami.
    \item Recenzje o telefonach komórkowych.
    \item Artykuły związane z telefonami komórkowymi.
    \item Forum wraz z opiniami użytkowników.
    \item Ocena użytkowników danego telefonu komórkowego.
    \item Opinie użytkowników na temat używanych telefonów komórkowych.
    \item Możliwość porównania telefonów komórkowych.
\end{enumerate}

\section{Dostępne rozwiązania}
W tej sekcji znajduje się spis dostępnych rozwiązań. Rozwiązania te zostaną dodatkowo porównane do rozwiązania autora, dzięki czemu możemy zobaczyć, czy dana strona sprosta wymaganiom użytkowników.

\subsection{GSM Arena}
\href{https://www.gsmarena.com/}{GSM Arena} \cite{gsm_arena} jest to anglojęzyczny portal dotyczący telefonów komórkowych. W tym serwisie możemy przeglądać przeróżne urządzenia. Na stronie znajdują się tam artykuły o zapowiedziany nowych urządzeniach. Serwis ten posiada również pobieranie cen z serwisów aukcyjnych takich jak: Amazon, AliExpress czy EBay.Rysunek \ref{GSM_Arena_1} przedstawia dane techniczne telefonu OnePlus 10 Pro wraz z jego aktualnymi cenami.
\begin{figure}[h]
    \centering
    \includegraphics[scale=0.38]{img/GSM Arena/DetailsPageWithPrices.png}
    \caption{Strona GSM Arena - Dane techniczne dla telefonu komórkowego: OnePlus 10 Pro}
    \label{GSM_Arena_1}
\end{figure}
Atutem tego serwisu są zdjęcia z danego telefonu, gdzie każdy użytkownik może sam ocenić zdjęcia w pełnej rozdzielczości. Jest to bardzo ważne dla amatorów fotografii. Rysunek \ref{GSM_Arena_2} przedstawia zdjęcia w dobrych warunkach oświetleniowych z OnePlus 10 Pro. 
\begin{figure}[H]
    \centering
    \includegraphics[scale=0.76]{img/GSM Arena/ReviewWithPhotos.png}
    \caption{GSM Arena - Recenzja telefonu komórkowego zdjęcia w dobrych warunkach oświetleniowych z OnePlus 10 Pro}
    \label{GSM_Arena_2}
\end{figure}
GSM Arena posiada zdjęcia w różnych warunkach oświetlenia, które są bardzo pomocne przy wyborze telefonu. Rysunek \ref{GSM_Arena_3} przedstawia zdjęcia z OnePlus 10 Pro w słabych warunkach oświetleniowych.
\begin{figure}[H]
    \centering
    \includegraphics[scale=0.7]{img/GSM Arena/GSMArena_picture_sample_lowlight.png}
    \caption{GSM Arena - Recenzja telefonu komórkowego: OnePlus 10 Pro}
    \label{GSM_Arena_3}
\end{figure}
Portal ten nie posiada natomiast fora, gdzie użytkownicy mogliby się wypowiadać na dany temat. GSM Arena pozwala na wystawienie opinii tekstowej. Brakuje możliwości oceny telefonu według kategorii jak czas pracy na baterii czy budowa urządzenia. Strona posiada tylko możliwość zapisania telefonu do listy ulubionych. Rysunek \ref{GSM_Arena_4} przedstawia najważniejsze dane techniczne wraz z możliwością dodania telefonu do listy ulubionych.

\begin{figure}[H]
    \centering
    \includegraphics[scale=0.7]{img/GSM Arena/GSMArena_become_a_fan.png}
    \caption{GSM Arena - Dodanie telefonu do listy ulubionych}
    \label{GSM_Arena_4}
\end{figure}

Na stronie są wyświetlane komentarze jednakże ich jakość jest marna. Nie wypowiadają się użytkownicy, którzy potwierdzili posiadanie danego urządzenia. Dla osoby kupującej używany telefon komórkowy, nie jest to korzystne, ponieważ nie może dowiedzieć się o mankamentach danego telefonu bądź też degradacji baterii. Na rysunku \ref{GSM_Arena_5} przedstawiono opinie użytkowników o OnePlus 10 Pro.
\begin{figure}[H]
    \centering
    \includegraphics[scale=0.52]{img/GSM Arena/Comments.png}
    \caption{GSM Arena - Opinie o OnePlus 10 Pro}
    \label{GSM_Arena_5}
\end{figure}

W tabeli \ref{comparison_gsm_arena} znajduje się porównanie rozwiązań, które zostały zauważone na stronie GSM Arena do rozwiązania idealnego. Widzimy jednak, że brakuje cech, które definiowałyby rozwiązanie idealne.
\begin{table}[H]
    \centering
        \begin{tabular}{|l|l|}
        \hline
        \multicolumn{1}{|c|}{Cecha rozwiązania}    & \multicolumn{1}{c|}{GSM Arena} \\ \hline
        Aktualne dane, z nadchodzącymi premierami telefonami & x                             \\ \hline
        Recenzje smartfonów                                  & x                                \\ \hline
        Forum wraz z opiniami użytkowników                   & -                                \\ \hline
        Ocena użytkowników danego smartfona                  & x                             \\ \hline
        Opinie użytkowników na temat używanych smartfonów    & -                                 \\ \hline
        Możliwość porównania smartfonów                      & x                                \\ \hline
        \end{tabular}
    \caption{Podsumowanie cech rozwiązania idealnego i GSM Areny -}
\label{comparison_gsm_arena}
\end{table}
W tabeli \ref{comparison_gsm_arena} zostały przyjęte oznaczenia: \textit{x} oznacza, że dane rozwiązanie ma pożądaną cechę, \textit{-} oznacza brak danej cechy w przedstawionym rozwiązaniu.
Podsumowując, GSM Arena jest to portal, który najbardziej przypomina rozwiązanie. Do rozwiązania brakuje mu cech, który umożliwiłyby kupno urządzeń używanych na stronie.

\subsection{mgsm.pl}
\href{https://www.mgsm.pl/pl/}{mgsm.pl} \cite{mgsm} jest to polskojęzyczny portal dotyczący smartfonów. Możemy na nim zobaczyć przeróżne testy telefonów komórkowych, dodatkowo są opakowane w formę tekstową oraz video. Strona ta posiada stronę z aktualnościami, gdzie umieszczane są aktualne pogłoski na temat nowych urządzeń czy też wiadomości dotyczące nowych premier telefonów komórkowych.
\begin{figure}[H]
    \centering
    \includegraphics[width=15cm]{img/mgsm/mgsm.png}
    \caption{Słowniczek przydatnych zwrotów znajdujący się na portalu mgsm.pl}
    \label{mgsm}
\end{figure}
Na stronie znajdują się przeróżne rankingi dotyczące popularności danego telefonu. Strona także udostępnia informację o tym ile wizyt ma dany smartfon jest to wygodne dla użytkownika, bo w prosty sposób pokazuje, który telefon miał w niedalekim czasie swoją premierę.
\begin{figure}[H]
    \centering
    \includegraphics[width=15cm]{img/mgsm/rankingsMgsm.png}
    \caption{Na stronie znajdują się rankingi wraz z łączną liczbą wizyt stronnie dziennie.}
    \label{mgsm_1}
\end{figure}
W dokładny i przejrzysty sposób są ukazane dane techniczne danych urządzeń. Co więcej w każdej ze specyfikacji telefonów są ukazane zdjęcia z danego urządzenia. W przypadku polskojęzycznych portali, nie dysponują połączeniem z zagranicznymi serwisami aukcyjnymi jak Amazon czy Ebay. Mamy tutaj dostęp tylko i wyłącznie do sklepu Media expert, więc możliwości tego portalu są ograniczone do jednego sklepu.
Brakuje także dostępu do polskich serwisów aukcyjnych takich jak: Allegro czy Olx, gdzie użytkownicy mogliby wystawiać ogłoszenia dotyczące sprzedaży ich użytkowników.
\begin{figure}[H]
    \centering
    \includegraphics[width=15cm]{img/mgsm/DetailsMgsm.png}
    \caption{Przykładowa strona wraz z danymi technicznymi dla urządzenia Samsung Galaxy S20FE}
    \label{mgsm_2}
\end{figure}
Kolejną zaletą tego portalu są bardziej obszerne opinie użytkowników na temat danego urządzenia. Są one oznaczane jako: pozytywne, negatywne lub neutralne. Dodatkowo użytkownicy opisują dokładnie zalety i wady danego urządzenia, które są oparte stycznością z danym urządzeniem.
\begin{figure}[H]
    \centering
    \includegraphics[width=15cm]{img/mgsm/mgsmComments.png}
    \caption{Przykładowy komentarz dotyczący urządzenia Samsung Galaxy S20FE}
    \label{mgsm_3}
\end{figure}
Inną zaletą tego portalu jest zawarty słowniczek dla osób, które nie mają do czynienia na co dzień z technologią. 
\begin{figure}[H]
    \centering
    \includegraphics[width=15cm]{img/mgsm/dictonary.png}
    \caption{Słowniczek przydatnych zwrotów znajdujący się na portalu mgsm.pl}
    \label{mgsm_4}
\end{figure}
W poniżej tabeli znajduje się porównanie rozwiązań, które zostały zauważone na stronie mgsm.pl do rozwiązania idealnego. Widzimy jednakże, że brakuje cech, które definiowałyby rozwiązanie idealne.
\begin{table}[H]
\centering
\begin{tabular}{|l|l|}
    \hline
    \multicolumn{1}{|c|}{Cecha rozwiązania idealnego}    & \multicolumn{1}{c|}{mgsm.pl} \\ \hline
    Aktualne dane, z nadchodzącymi premierami telefonami & x                             \\ \hline
    Recenzje smartfonów                                  & x                                \\ \hline
    Forum wraz z opiniami użytkowników                   & -                                \\ \hline
    Ocena użytkowników danego smartfona                  & x                             \\ \hline
    Opinie użytkowników na temat używanych smartfonów    & -                                 \\ \hline
    Możliwość porównania smartfonów                      & x                                \\ \hline
\end{tabular}
\caption{Podsumowanie cech rozwiązania idealnego i msgm - \textit{x} oznacza, że dane rozwiązanie ma pożądaną cechę, \textit{-} oznacza brak danej cechy w przedstawionym rozwiązaniu.}
\label{comparison_mgsm}
\end{table}
Podsumowując mgsm.pl jest to rozwiązanie, które przypomina rozwiązanie idealne jednakże brakuje mu kilku cech, które ma rozwiązanie idealne.

\subsection{Kimovil}
\href{https://www.kimovil.com/pl/}{Kimovil} \cite{kimovil} jest to wielojęzyczna przeglądarka sprzętu elektronicznego, na stronie znajdują się także inny sprzęt elektroniczny taki jak telewizory czy tablety. Dodatkowym atutem strony są zniżki czy promocje na zakup sprzętu elektronicznego. Strona udostępnia także newsletter, aby móc obserwować różne promocje na dany sprzęt elektroniczny.
\begin{figure}[H]
    \centering
    \includegraphics[width=15cm]{img/Kimovil/kimovil.png}
    \caption{Strona główna portalu Kimovil}
    \label{kimovil_1}
\end{figure}
Na stronie są zamieszczone również rankingi, gdzie użytkownicy mogą oznaczać urządzenia jako urządzenie, które pożądają. Dodatkowym rankingiem jest ranking smartfonów, które użytkownicy posiadają lub posiadali w przeszłości. 
\begin{figure}[H]
    \centering
    \includegraphics[width=15cm]{img/Kimovil/rankingsKimovil.png}
    \caption{Rankingi na stronie Kimovil}
    \label{kimovil_2}
\end{figure}
Na stronie znajdują się dane techniczne danego urządzenia, które w przejrzysty sposób pokazuję stosunek ceny do jakości według serwisu Kimovil. Co więcej, można zauważyć, że przy danych technicznych mamy wymienione sklepy wraz z najkorzystniejszą ofertą dla danego telefonu. Przy danych technicznych mamy także dostępne podsumowanie, gdzie ukazana jest np. nazwa matrycy oraz procesor, która została użyta w danym telefonie. Ważną informacją dla użytkownika jest także informacja o używanych pasm sieci telekomunikacyjnych z rozróżnieniem na sieci. Jest to szczególnie przydatne w Polsce, ze względu na obsługiwanie danych pasm sieci przez danych operatorów sieci komórkowych.
\begin{figure}[H]
    \centering
    \includegraphics[width=15cm]{img/Kimovil/kimovilDetails.png}
    \caption{Przykładowa strona z danymi technicznymi w serwisie Kimovil (Xiaomi Redmi K50 Gaming Edition)}
    \label{kimovil_3}
\end{figure}
Dla każdego smartfona udostępniane są przez użytkowników opinie. Każda opinia czy też komentarz są udzielone wraz z oceną danej komponentu smartfona. Znajdują się tam dodatkowe informację dotyczące czasu pracy na baterii lub też informacja ile dany użytkownik posiada dany sprzęt. Dodatkowo każdy telefon oraz każdy jego komponent jest oceniany w skali od 1-10.
\begin{figure}[H]
    \centering
    \includegraphics[width=15cm]{img/Kimovil/opinieKimovil.png}
    \caption{Przykładowa strona z opiniami w serwisie Kimovil (Xiaomi Redmi K50 Gaming Edition)}
    \label{kimovil_4}
\end{figure}
W poniższej tabeli znajduje się porównanie rozwiązań, które zostały zauważone na stronie Kimovil do rozwiązania idealnego. Widzimy jednakże, że brakuje cech, które definiowałyby rozwiązanie idealne.
\begin{table}[H]
\centering
\begin{tabular}{|l|l|}
    \hline
    \multicolumn{1}{|c|}{Cecha rozwiązania idealnego}    & \multicolumn{1}{c|}{Kimovil} \\ \hline
    Aktualne dane, z nadchodzącymi premierami telefonami & x                             \\ \hline
    Recenzje smartfonów                                  & -                                \\ \hline
    Forum wraz z opiniami użytkowników                   & -                                \\ \hline
    Ocena użytkowników danego smartfona                  & x                             \\ \hline
    Opinie użytkowników na temat używanych smartfonów    & x                                 \\ \hline
    Możliwość porównania smartfonów                      & x                                \\ \hline
\end{tabular}
\caption{Podsumowanie cech rozwiązania idealnego i Kimovil - \textit{x} oznacza, że dane rozwiązanie ma pożądaną cechę, \textit{-} oznacza brak danej cechy w przedstawionym rozwiązaniu.}
\label{comparison_kimovil}
\end{table}
Podsumowując, Kimovil jest to rozwiązanie, które przypomina rozwiązanie idealne jednakże brakuje mu kilku cech, które ma rozwiązanie idealne. Dodatkowym atutem jest posiadanie wielu języków wraz z odpowiadającym do kraju walutami.

\subsection{Versus}
Versus \cite{versus} jest wielojęzyczną porównywarką. Dodatkowym atutem tej porównywarki tak samo jak w przypadku Kimovil, ma możliwość porównywania innego sprzętu elektronicznego. Na stronie również można na podstawie opinii użytkowników porównać także miasta czy uczelnie. 
Na tym portalu znajduje się duża baza telefonów, jednakże dane są uzupełniane przez użytkowników. Takie powoduje, że dane dotyczące mogą się różnić od danych, które dostarcza producent.
\begin{figure}[H]
    \centering
    \includegraphics[width=15cm]{img/versus/versusRankingi.png}
    \caption{Rankingi znajdujące się na stronie versus}
    \label{versus_1}
\end{figure}
Na stronie ukazane jest specyfikacja danych urządzeń, w postaci podsumowania. Dane dotyczące specyfikacji pochodzą od użytkowników. Użytkownik po wejściu na stronę danego urządzenia jest witany podsumowaniem dotyczący danego urządzenia. W tym podsumowaniu możemy zobaczyć najważniejsze dane dotyczące telefonu tj.: aparat, ekran czy pamięć RAM.
\begin{figure}[H]
    \centering
    \includegraphics[width=15cm]{img/versus/versusDetails.png}
    \caption{Podsumowanie danych dotyczące telefonu Samsung Galaxy S22 Ultra}
    \label{versus_2}
\end{figure}
Przy każdym urządzeniu oprócz tego umieszczone są opinie użytkowników. Opinie te dzielą się na poszczególne komponenty danego urządzenia. Każdy użytkownik, ocenia dany komponent w skali od 1-10.
Co więcej, każdy komentarz ma osobisty komentarz użytkownika, w którym dany użytkownik wymienia wady i zalety oraz problemy, które napotkał z danym urządzeniem.
\begin{figure}[H]
    \centering
    \includegraphics[width=15cm]{img/versus/versusOpinie.png}
    \caption{Ukazanie średniej wartości poszczególnych kategorii dla urządzenia - Samsung Galaxy S22 Ultra}
    \label{versus_3}
\end{figure}
\begin{figure}[H]
    \centering
    \includegraphics[width=15cm]{img/versus/versusKomentarze.png}
    \caption{Komentarz nt. urządzenia Samsung Galaxy S22 Ultra}
    \label{versus_4}
\end{figure}
W poniżej tabeli znajduje się porównanie rozwiązań, które zostały zauważone na stronie Versus do rozwiązania idealnego. Widzimy jednakże, że brakuje cech, które definiowałyby rozwiązanie idealne.
\begin{table}[H]
    \centering
    \begin{tabular}{|l|l|}
        \hline
        \multicolumn{1}{|c|}{Cecha rozwiązania idealnego}    & \multicolumn{1}{c|}{Versus} \\ \hline
        Aktualne dane, z nadchodzącymi premierami telefonami & x                             \\ \hline
        Recenzje smartfonów                                  & -                                \\ \hline
        Forum wraz z opiniami użytkowników                   & -                                \\ \hline
        Ocena użytkowników danego smartfona                  & x                             \\ \hline
        Opinie użytkowników na temat używanych smartfonów    & -                                \\ \hline
        Możliwość porównania smartfonów                      & x                              \\ \hline
    \end{tabular}
    \caption{Podsumowanie cech rozwiązania idealnego i Versus - \textit{x} oznacza, że dane rozwiązanie ma pożądaną cechę, \textit{-} oznacza brak danej cechy w przedstawionym rozwiązaniu.}
    \label{comparison_versus}
\end{table}
Versus jest to rozwiązanie, które głównie ma spełniać funkcjonalność porównywarki. Brakuje mu kilku cech, które posiadałoby rozwiązanie idealne. Dużą wadą tego serwisu jest uzupełnianie przez użytkowników dane, które nie zawsze okazują się prawdziwe.
