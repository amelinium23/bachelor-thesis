\section{Założenia projektowe}
W tym rozdziale zostały zawarte wszystkie wymagania funkcjonalne oraz niefunkcjonalne aplikacji. Projekt posiada funkcjonalności i założenia, które są zawarte w rozdziale \ref*{ideal_solution}. 
\subsection{Wymagania funkcjonalne}
\subsubsection{Użytkownik niezalogowany}
\begin{enumerate}
    \item Użytkownik posiada możliwość zalogowania się za pomocą e-maila i hasła.
    \item Użytkownik posiada możliwość przeglądania oraz sortowania marek telefonów komórkowych.
    \item Użytkownik posiada możliwość przeglądania oraz sortowania telefonów komórkowych.
    \item Użytkownik posiada możliwość przeglądania najnowszych postów z forum.
    \item Użytkownik posiada możliwość porównywania maksymalnie trzech urządzeń znajdujących się na stronie.
    \item Użytkownik posiada możliwość przeglądania cen dla wybranego modelu telefonu komórkowego.
\end{enumerate}
\subsubsection{Użytkownik zalogowany}
\begin{enumerate}
    \item Użytkownik zalogowany musi mieć możliwość wykonywania wszystkich czynności użytkownika niezalogowanego.
    \item Użytkownik posiada możliwość weryfikacji adresu mail.
    \item Użytkownik posiada możliwość edycji, dodawania swojego aktualnego telefonu.
    \item Użytkownik posiada możliwość edycji swojego profilu.
    \item Użytkownik posiada możliwość dodawania, edycji oraz usuwania swoich postów.
    \item Użytkownik posiada możliwość przeglądania oraz sortowania swoich postów.
    \item Użytkownik posiada możliwość reagowania na posty.
    \item Użytkownik posiada możliwość przeglądania oraz dodawania opinii dotyczących urządzeń.
\end{enumerate}

\subsection{Wymagania niefunkcjonalne}
\begin{enumerate}
    \item System łączy się z zewnętrznym serwisem (API GSM Arena) w celu pozyskania danych dotyczących telefonów komórkowych.
    \item System ma możliwość zapisywania i odczytywania danych znajdujących się w bazie danych.
    \item System musi przechowywać dane na temat użytkownika, tworzącego opinie dotyczące urządzeń.
\end{enumerate}

\newpage
Na rysunku \ref*{use_case} zostały przedstawione przypadki użycia aplikacji.
\begin{figure}[H]
    \centering
    \includegraphics[scale=0.41]{diagrams/useCaseDiagram.png}
    \caption{Diagram przypadków użycia}
    \label{use_case}
\end{figure}


\subsection{Użyte technologie}
\subsubsection{Frontend}
Część frontendowa została zbudowana w oparciu o język TypeScript \cite{TypeScript}. TypeScript jest to język opracowany przez firmę Microsoft, który pozwala na wprowadzenie statycznego typowania do JavaScript. Wadą tego języka jest kompilowanie go do JavaScriptu, co wydłuża czas budowania aplikacji. Dodatkowo, dzięki typom i integracji IDE występuje możliwość lepszego podpowiadania, jakie własności posiada dany obiekt.
Język ten został wybrany ze względu na tworzenia własnych typów, aby móc odwzorować typy, które zostały zdefiniowane w projekcie. Dzięki stworzonym typom, usprawnia pracę programisty oraz zwiększa czytelność kodu. 

W warstwy widoku została zastosowana biblioteka React \cite{React}. React jest to biblioteka do warstwy widoku, która pozwala na tworzenie komponentów w celu zbudowania widoku. Biblioteka ta charakteryzuje się dużą wydajnością dzięki zastosowania mechanizmu Virtual DOM, który pozwala na szybsze porównywanie zmian pomiędzy faktycznymi elementami na stronie. Pozwala ona na tworzenie komponentów, które będziemy mogli wielokrotnie używać danego komponentu w wielu miejscach strony. Do poruszania się po stronie w projekcie została użyta biblioteka React router \cite{react_router}, która umożliwia na zdefiniowanie podstron oraz nawigację pomiędzy nimi. Do umieszczania powiadomień na stronie została użyta biblioteka react toastify \cite{react_toastify}.

Wykorzystaną biblioteką z komponentami UI jest React Bootstrap. Biblioteka ta pozwala na użycie gotowych komponentów, które zostały stylowane przez twórców frameworka Bootstrap. Framework ten jest standardem dla aplikacji webowych, co spowodowało zbudowanie wielu innych frameworków opartych na Bootstrapie. jest dużo Ta biblioteka została wybrana ze względu, na prostotę jej użycia oraz wyglądu komponentów. 

\subsubsection{Backend}
Część backendowa została oparta na języku interpretowanym Python \cite{python}. Python jest to język opracowywany przez Python Software Foundation \cite{psf}, który pozwala na tworzenie aplikacji webowych czy prowadzenia analiz statystycznych. Język ten jest uniwersalny, dzięki któremu mamy możliwość zbudowania aplikacji webowych, konsolowych czy tworzenia sieci neuronowych. Wadą tego języka jest jego wydajność, wynika to z faktu działania interpretera, który musi skonwertować program na kod bajtowy, który następnie jest odczytywany i uruchamiany na maszynie wirtualnej Python'a. Zaletą tego języka jest łatwość budowania aplikacji webowych wraz z składnią języka i systemem wcięć. Do projektu został wybrany za jego prostotę użycia wraz z możliwością do analizy statystycznej.

Frameworkiem, który został użyty do zbudowania API jest jeden z najpopularniejszych frameworków Flask \cite{flask}. Flask jest to framework, cechującym się dużą wydajnością. Posiada on dużą możliwość rozbudowy poprzez pluginy, dzięki którym mamy możliwość łączenia się z bazami danych czy umożliwieniem autoryzacji. W tym projekcie Flask, został rozszerzony o bibliotekę Flask-APScheduler \cite{flask_apscheduler}. Plugin ten pozwala na zaplanowanie czynności w czasie, dzięki czemu jest możliwe cogodzinne wpisywanie danych telefonów do bazy danych. API pozwala na autoryzację użytkowników przy pomocy Firebase \cite{firebase}, dodawaniem zdjęć do Cloud Storage \cite{cloud_storage} i dodawaniem urządzeń użytkownika oraz postów do bazy danych Firestore \cite{firestore}. Flask został wybrany do projektu ze względu na możliwość związaną z rozszerzaniami oraz wydajność.

\subsection{Użyte narzędzia CI [\ref{CI}] i CD [\ref{CD}]}
W projekcie zdecydowałem się na użycie trzech narzędzi CI/CD. Ze względu na rozdzielenie części frontendowej i backendowej na dwa repozytoria, pozwoliło to na łatwiejsze zastosowanie Github Actions. Github Actions \cite{github_actions} jest to funkcja, które pozwala na przeprowadzenie testów, zbudowanie i wdrożenie aplikacji do publicznego użycia. Dzięki tej funkcji jesteśmy w stanie sprawdzać stan budowania aplikacji czy zobaczyć raport wykonania testów.

Dla części backendowej został stworzony akcja za pomocą Github Actions, której zadaniem jest wdrożenie aplikacji na hosting render \cite{render}. Jest to jeden z wiodących hostingów, dla aplikacji webowych stworzonych w języku Python. Wadą tego serwisu jest współdzielenie mocy obliczeniowej na darmowym poziomie subskrypcji. Zaletą tego hostingu jest integracja z portalem GitHub, dzięki czemu za pomocą jednego kliknięcia mamy dostęp do kodu aplikacji.

Dla repozytorium z częścią frontendowa wykonano integrację z hostingiem Vercel. Vercel \cite{vercel} jest to hosting, który pozwala na wprowadzenie aplikacji frontendowych. Hosting ten charakteryzuje się dużą ilością szablonów dzięki któremu nie martwimy się o specjalną konfigurację dla danej aplikacji. Kolejną możliwością tego hostingu jest darmowe przypisanie własnych domen, co jest jednocześnie wadą. Posiadając darmową subskrypcję jesteśmy ograniczeni do domen kończących się na \textit{vercel.app} Hosting ten posiada integrację z wieloma systemami kontroli wersji: Github, Gitlab czy Bitbucket. Dodatkową opcją jest optymalizacja zdjęć, która posiada hosting, co pozwala na ograniczeniu pobierania obrazów na stronie.

\section{Architektura aplikacji}
W tej sekcji znajdują się opisane jak zostały podzielone moduły aplikacji zarówno części frontendowej oraz backendowej. Opisane także zostały wybrane źródła danych, które zostały wykorzystane w projekcie. Zawarty został również opis wraz z diagramem encji.

\subsection{Źródła danych}\label{data_sources}
W projekcie zostały wybrane dwa źródła danych. Zostały one wybrane także, ze względu na ich wysoką dostępność w internecie. Podstawowym źródłem danych jest GSM Arena API, dzięki czemu mogliśmy uzyskać dane dotyczące telefonów komórkowych. Dodatkowym źródłem danych są serwisy aukcyjne, które dostarczają aktualne ceny telefonów komórkowych.

\subsubsection{Źródło danych o telefonach}
Źródłem danych o telefonach jest GSM Arena API \cite{gsm_arena_api}, stworzone przez Kyaw Htut. Dostarcza one dane dotyczące urządzeń znajdujących się na portalu GSM Arena \cite{gsm_arena}. API wykorzystuję metodę Web Scraping, dzięki któremu otrzymujemy dane w formacie JSON.

Wadą tego rozwiązania jest umieszczenie skryptu na serwisie Google App Script \cite{google_app_script}. Zostało to wymuszone, z powodu dodatkowej biblioteki Cheerio, które wykorzystuję API. Dodatkowo Google App Script posiada ograniczoną liczbę możliwych wywołań API. Warto zaznaczyć, że API nie jest oficjalnie udostępnione przez GSM Arena, dlatego nie jestem w stanie zapewnić, że będzie ono działało w przyszłości. Sam autor dostarcza pełną dokumentację API za pomocą programu Postman. W dokumentacji \cite{gsm_arena_api_reference} zostały zawarte przykładowe żądania, co pozwala na łatwe wykorzystanie API.

\subsubsection{Serwisy aukcyjne}
Drugim źródłem danych są serwisy aukcyjne. W projekcie zostały wykorzystane serwisy jak: Amazon, Ebay, AliExpress oraz sklepy producentów. Te dane zostały udostępnione w API Gsm Areny, nie było więc konieczności ich pobierania z serwisów.

Początkowo w projekcie, miały znaleźć się dane serwisów zagranicznych miały zostać rozszerzone o dane z polskich serwisów aukcyjnych takich jak: Allegro, Olx, Skąpiec czy Ceneo. Serwisy te, aby pobrać dane, pobierają dodatkową opłatę, co więcej należy spełnić ich warunki, aby móc korzystać z API.

W przypadku serwisu Allegro, należy być w programie partnerskim oraz aplikacja w Allegro musi być zweryfikowana. Weryfikacja odbywa się po skontaktowaniu się z Allegro. Aby otrzymać weryfikację, należy mieć 100 aktywnych użytkowników, zarejestrowanych w aplikacji oraz być w programie partnerskim. Na rysunku \ref*{allegro_verification} została przedstawione warunki weryfikacji aplikacji.

\begin{figure}[H]
    \centering
    \includegraphics[scale=0.55]{responses/allegro_response.png}
    \caption{Odpowiedź Allegro na prośbę o weryfikację aplikacji}
    \label{allegro_verification}
\end{figure}

Zatem integracja z tym serwisem nie była możliwa. Podobny sposób działania mają także serwisy Skąpiec \cite{skapiec} i Ceneo \cite{ceneo}. Również wymagają dołączenie do programu partnerskiego, aby móc pobierać oferty z serwisu.

Innym przypadkiem jest serwis OLX. Ten serwis nie udostępnia możliwości pobierania ofert publicznych, a jedynie ofert zalogowanego użytkownika. Na rysunku \ref*{olx_response} została przedstawiona odpowiedź serwisu na prośbę o udostępnienie API.

\begin{figure}[H]
    \centering
    \includegraphics[scale=0.58]{responses/olx_response.png}
    \caption{Odpowiedź Allegro na prośbę o weryfikację aplikacji}
    \label{olx_response}
\end{figure}

W rezultacie, w projekcie zostały jedyne dane z serwisów aukcyjnych, które nie wymagały dodatkowej weryfikacji czy też opłaty.

\subsection{Wybrana baza danych}
W projekcie została wykorzystana NoSQL \cite{NoSQL} baza danych Firestore \cite{firestore}. Baza ta została zbudowana w paradygmacie dokumentowym, co pozwala na łatwe zapisywanie danych. Zaletą takiego rozwiązania jest brak określenia dokładnej struktury danych. W przypadku zmiany typu danego pola, nie musimy przeprowadzać migracji bazy danych, tylko możemy dowolnie zmienić strukturę danych. W przypadku tego projektu jest to dobre rozwiązanie, ze względu na pobieranie danych z zewnętrznych źródeł, które mogą się zmieniać.

Dzięki paradygmatowi dokumentowemu, baza danych została podzielona na kolekcję:
\begin{enumerate}
    \item \textit{brands} - przechowuje dane o producentach telefonów. Kolekcja, ta została stworzona, jako kopia zapasowa danych, pobieranych z GSM Areny. W przypadku, gdyby serwis nie był dostępny, można by w łatwy sposób zastąpić dane z bazy danych. Zawiera ona jeden dokument, który składa się z listy producentów, wraz z ich nazwą i identyfikatorem.
    \item \textit{devices} - przechowuje dane o telefonach. Tak samo jak w przypadku producentów, należy także synchronizować dane wraz z producentami pobranymi z GSM Areny. Zawiera jeden dokument, który składa się z pogrupowanej listy telefonów, wraz z ich nazwą, identyfikatorem, zdjęciem w postaci linku oraz klucza, który pozwala otrzymać specyfikację danego urządzenia.
    \item \textit{device} - przechowuje dane o telefonie użytkownika. Zawiera ona dokumenty, które są nazywane identyfikatorem użytkownika, w celu uniknięcia duplikacji danych. W dokumentach znajdują się dane o telefonie użytkownika, wraz z jego nazwą, zdjęciem oraz kluczem, który pozwala otrzymać specyfikację danego urządzenia.
    \item \textit{opinions} - przechowuje opinie dotyczące danego urządzenia. Zawiera ona dokumenty, które są nazywane identyfikatorem urządzenia, w celu uniknięcia duplikacji danych. W dokumentach znajdują się dane o opinii, wraz z jej treścią, tytułem, identyfikatorem oraz informacją o użytkowniku, który wystawił daną opinie.
    \item \textit{posts} - przechowuje posty użytkowników. Zawiera ona dokumenty, które są nazywane identyfikatorem użytkownika, w celu uniknięcia duplikacji danych. W dokumentach znajdują się dane o postach, wraz z treścią, tytułem, identyfikatorem, komentarzami oraz informacją o użytkowniku, który wystawił daną opinie. W przypadku, gdy użytkownik, wystawił prośbę o wycenę telefonu, wtedy użytkownik może dodać zdjęcia telefonu. W bazie są przechowywane jako linki, do zdjęć znajdujące się na Google Cloud Storage.
\end{enumerate}

W tej bazie można wyróżnić kilka struktur danych. Można postrzegać je jako encję w rozumieniu relacyjnej bazy danych. 
Na rysunku \ref{database} przedstawiono schemat struktur danych w bazie danych.

\begin{figure}[H]
    \centering
    \includegraphics[scale=0.46]{diagrams/entityDiagram.png}
    \caption{Schemat struktur danych w bazie danych}
    \label{database}
\end{figure}

\subsection{Schemat modułów aplikacji}
\subsubsection{Backend}

\subsubsection{Frontend}


\subsection{Zastosowane techniki pozyskiwania informacji}
