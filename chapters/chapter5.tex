Cel pracy został osiągnięty, aplikacja została wyposażona we wszystkie założone w rozdziale \ref{ideal_solution} funkcjonalności. Po wykonaniu aplikacji zostały sformułowane problemy, które można poprawić w ramach dalszego rozwijania aplikacji. Warto je podsumować i zaplanować dalsze kroki. Plany związane z naprawą i rozwijaniem aplikacji zostały opisane w tym rozdziale.

Jednym z aspektów nadających się do poprawy jest wygląd aplikacji.\linebreak\textit{Bootstrap} \cite{bootstrap} jest to framework, który pozwala na szybkie zbudowanie aplikacji, ale nie jest on idealny. Warto byłoby zastanowić się nad wyglądem aplikacji, aby była bardziej funkcjonalna dla użytkownika. Dużo do życzenia pozostawia całość wrażeń z użytkowania aplikacji. Wartościowa byłaby poprawa walidacji pól w formularzach, aby użytkownik miał lepszą wiedzę o tym, co jest niepoprawne. Ponadto zasadne byłoby poprawić zarządzanie formularzami poprzez dodanie biblioteki do zarządzania nimi. Przykładami takich bibliotek jest \textit{Formik} \cite{Formik} lub \linebreak \textit{React Hook Form} \cite{React_hook_form}. 

Autor również zauważył możliwość zastosowania innych technologii, które mogą poprawić wydajność aplikacji. Pierwszym etapem takich działań może być zastąpienie frameworka CSS na bardziej przyjazny dla użytkownika. W tym celu autor wykorzystałby framework \textit{Mantine} \cite{mantine}, który posiada wbudowaną możliwość zmiany motywu oraz jego mały rozmiar pozwoliłby także na zwiększenie wydajności aplikacji. Dodatkowo autor zastosowałby bibliotekę \textit{Tanstack-Query} (dawnej \textit{React-Query}) \cite{tanstack_query}, która pozwala na zastosowanie techniki \textit{Infinite Query}, która pozwoliłaby na zaimplementowanie \textit{Infinite Scroll} \cite{infinite_scroll} w przypadku listy telefonów oraz producentów.

Dodatkowo w celu poprawienia wydajności części frontendowej, autor pracy rozważyłby także inne technologie. Przykładem takiej technologii jest \textit{NextJS} \cite{nextjs}, który pozwala na szybsze budowanie aplikacji. Framework ten pozwoliłby na szybszy proces tworzenia i rozwijania aplikacji. Dodatkowo umożliwia łatwiejsze przeprowadzenie mutacji danych na stronie, bez konieczności przeładowania strony. Zaletą tej technologii jest także \textit{Server Side Rendering} \cite{SSR}, który pozwala na szybsze ładowanie strony, gdyż nie jest ona generowana po stronie klienta. Dodatkowo w przypadku tej aplikacji, \textit{NextJS} \cite{nextjs} udostępnia możliwość użycia metody \textit{lazy loading} na zdjęciach, co pozwoliłoby na szybsze ładowanie strony.

Serwer aplikacji został zbudowany na podstawie frameworka \textit{Flask} \cite{flask}. Spowodowało to, że aplikacja jest bardzo łatwo modyfikowalna. Jednakże nie jest on idealny, gdyż jego główną wadą jest jego wydajność. W zastosowaniu przy tak dużej ilości danych, wydajność wymaga poprawy. Autor rozważyłby także zastosowanie serwera \textit{GraphQL}, który pozwoliłby na szybsze pobieranie danych z serwera. Pozwala to na łatwe modyfikowanie danych w aplikacji, bez konieczności pobierania wszystkich danych z serwera.

Dodatkowo autor rozważyłby zmianę języka i frameworka użytego w celu zbudowania serwera. Aktualnie na rynku występuje kilka frameworków, które pozwalają na zbudowanie go. Jednym z nich jest \textit{ExpressJs} \cite{express}, który pozwala na zbudowanie go w szybki sposób. Ze zmianą frameworka \textit{Flask} na \textit{ExpressJs} \cite{express} zostałby zmieniony język programowania na \textit{TypeScript} \cite{TypeScript}. Podobnie jak \textit{Flask} \cite{flask},\linebreak \textit{ExpressJs} \cite{express} jest to framework oparty na rozszerzeniach, co pozwala na rozwijanie aplikacji w bardzo szybkim tempie.

Autor zauważył również potencjał biznesowy aplikacji, dzięki któremu ograniczenia związane z darmowymi planami \textit{Firebase} \cite{firebase} oraz \textit{Render} \cite{render}, mogłyby zostać rozwiązane poprzez wykupienie odpowiedniej subskrypcji.

Warto zadbać o aspekt biznesowy aplikacji. Otwierając taką stronę należy zapłacić duże koszty jej utrzymania na serwerach Google. Jeżeli strona miałaby dziennie więcej niż 20 tysięcy odczytów telefonów, to koszty aplikacji rosłyby proporcjonalnie. W celu zadbania także o dodatkową ochronę użytkowników należałoby dodać weryfikację dwuetapową, zwiększa to bezpieczeństwo użytkownika na stronie. Jest to oczywiście dodatkowa płatność, jeżeli chcemy to wykonać za pomocą usług Google. W celu zarobienia na dodatkowe usługi można wprowadzić reklamy z Google AdSense, które mogą zwiększyć dochody z aplikacji.

Do aplikacji można wprowadzić możliwość pisania artykułów bądź felietonów, przez co pasjonaci danego urządzenia mogliby napisać artykuł na jego temat. Dodatkowo użytkownicy mogliby dowiedzieć się o najnowszych trendach w urządzeniach mobilnych. Wprowadzenie możliwości pisania artykułów powoduje także dodanie panelu administratora, który pozwoliłby na zarządzanie artykułami. Nie mamy też wpływu na treści artykułów, które mogą być wulgarne, więc należy zastosować dodatkowe zabezpieczenia, które uniemożliwią dodanie takich treści. Dzięki wprowadzeniu takiej funkcjonalności, aplikacja może stać się bardziej popularna, co zwiększyłoby przychody z reklam.

W aplikacji dodatkowo należałoby wprowadzić ograniczenia dotyczące liczby zdjęć przechowywanych na serwerze. Aktualnie aplikacja nie posiada takich ograniczeń, co może wiązać się z dużym zużyciem pamięci, co przekształca się na wyższe koszty utrzymania aplikacji. Wprowadzenie takich ograniczeń pozwoliłoby na zmniejszenie kosztów utrzymania aplikacji.
