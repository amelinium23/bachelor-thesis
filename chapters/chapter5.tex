\section{Podsumowanie}\label{summary}
Cel pracy został osiągnięty, aplikacja została wyposażona we wszystkie założone w rozdziale \ref{ideal_solution} funkcjonalności. Po wykonaniu aplikacji zostały sformułowane problemy, które można poprawić w ramach dalszego rozwijania aplikacji. Warto je podsumować i zaplanować dalsze kroki. Plany związane z naprawą i rozwijaniem aplikacji zostały opisane w rozdziale \ref{future_plans}.

Jednym z aspektów nadających się do poprawy jest wygląd aplikacji. Bootstrap jest to framework, który pozwala na szybkie zbudowanie aplikacji, ale nie jest on idealny. Warto byłoby zastanowić się nad wyglądem aplikacji, aby była bardziej funkcjonalna dla użytkownika. W przypadku wymiany frameworka, należałoby również zastanowić się nad wyglądem aplikacji. Dużo do życzenia pozostawia całość wrażeń z użytkowania aplikacji. Wartościowa byłaby poprawa walidacji pól w formularzach, aby użytkownik miał lepszą wiedzę o tym, co jest niepoprawne. Ponadto zasadne byłoby poprawić zarządzanie formularzami poprzez dodanie biblioteki do zarządzania formularzami. Przykładami takich bibliotek jest \textit{Formik} \cite{Formik} lub \textit{React Hook Form} \cite{React_hook_form}.

Autor pracy, rozważyłby także inne technologie, które mogłyby być użyte w aplikacji. Przykładem takiej technologii jest NextJS, który pozwala na szybsze budowanie aplikacji. Dodatkowo umożliwia łatwiejsze przeprowadzenie mutacji danych na stronie, bez konieczności przeładowania strony. Zaletą tej technologii jest także \textit{Server Side Rendering} \cite{SSR}, który pozwala na szybsze ładowanie strony, gdyż nie jest ona generowana po stronie klienta.

Serwer aplikacji został zbudowany na podstawie frameworka Flask. Spowodowało to, że aplikacja jest bardzo łatwo modyfikowalna. Jednakże nie jest on idealny, gdyż jego główną wadą jest jego wydajność. W zastosowaniu przy tak dużej ilości danych, wydajność wymagałaby poprawy. Autor rozważyłby także zastosowanie serwera GraphQL, który pozwoliłby na szybsze pobieranie danych z serwera. Pozwala to na łatwe modyfikowanie danych w aplikacji, bez konieczności pobierania wszystkich danych z serwera.

\section{Wnioski}
Po wykonaniu i przeanalizowaniu aplikacji, autor pracy doszedł do następujących wniosków:
\begin{enumerate}
  \item Wydajność części backendowej wymaga zmian w kierunku jej zwiększenia, należałoby poprawić wydajność serwera, aby aplikacja była bardziej przyjazna dla użytkownika. Wydajność części backendowej jest również zależna od wydajności serwisu \cite{render}, który został wykorzystany do upublicznienia aplikacji.
  \item Zastosowanie innej metody przechowywania w bazie danych informacji na temat urządzeń, producentów. Przez pobieranie danych telefonów z zewnętrznego opóźnia działanie aplikacji.
  \item Dodanie biblioteki do zarządzania formularzami, aby poprawić zarządzanie formularzami. Poprawiłoby to także pokazywanie błędów w formularzach, ze względu na wbudowaną walidację pól w formularzach.
  \item Wygląd aplikacji oraz wrażenia z jej użytkowania pozostawiają dużo do życzenia. Można rozważyć zmianę frameworka na bardziej przyjazny dla użytkownika. Przykładem takich frameworków są: TailwindCSS \cite{tailwindcss} oraz Mantine \cite{mantine}. Za pomocą tych frameworków można zbudować lepszy wygląd aplikacji.
  \item Do wczytywania danych można zastosować komponenty za Infinite Scroll, tak aby ładowanie dodatkowych telefonów było bardziej przyjazne dla użytkownika. Można również zastosować technikę Optimistic UI, które poprawiłaby wygląd aplikacji.
  \item Rozważenie zastosowania innych technologii, w celu poprawy wydajności aplikacji. Wykorzystanie NextJS \cite{nextjs}, który buduje statycznie strony oraz lepiej optymalizuje zarządzanie zasobami takimi jak: zdjęcia czy arkusze stylów.
  \item Przystosowanie aplikacji do urządzeń mobilnych, aby użytkownik korzystający z urządzenia mobilnego mógł skorzystać ze strony w tak samo wydajny sposób jak na przeglądarce.
  \item Zamiana technologii, w której został napisany serwer aplikacji oraz zrezygnowanie z języka Pythona. Ze względu na wydajność języka Python, autor rozważyłby zastosowanie języka TypeScript w części backendowej.
\end{enumerate}

\section{Plany na przyszłość}\label{future_plans}
W tej sekcji są przedstawione plany związane z rozwojem aplikacji. Autor pracy planuje dalszy rozwój aplikacji, w celu poprawienia jej jakości oraz wydajności. Autor  zauważył również potencjał biznesowy aplikacji, dzięki któremu ograniczenia związane z darmowymi planami Firebase oraz Render, mogłyby zostać rozwiązane poprzez wykupienia odpowiedniej subskrypcji.  

Warto zadbać o aspekt biznesowy aplikacji. Otwierając taką stronę należy zapłacić duże koszty za jej utrzymanie na serwerach Google. Jeżeli strona miałaby dziennie więcej niż 20 tysięcy odczytów telefonów, to koszty aplikacji rosłyby proporcjonalnie. W celu zadbania także o dodatkową ochronę użytkowników należałoby dodać weryfikację dwuetapową, zwiększa to bezpieczeństwo użytkownika na stronie. Jest to oczywiście dodatkowa płatność, jeżeli chcemy to wykonać za pomocą usług Google. W celu zarobienia na dodatkowe usługi można wprowadzić reklamy z Google AdSense, które mogą zwiększyć dochody z aplikacji.

Do aplikacji można wprowadzić możliwość pisania artykułów bądź felietonów, przez co pasjonaci danego urządzenia mogliby napisać artykuł na jego temat. Dodatkowo użytkownicy mogliby dowiedzieć się o najnowszych trendach w urządzeniach mobilnych. Wprowadzenie możliwości pisania artykułów powoduje także dodanie panelu administratora, który pozwoliłby na zarządzanie artykułami. Nie mamy też wpływu na treści artykułów, które mogą być wulgarne, więc należy zastosować dodatkowe zabezpieczenia, które uniemożliwią dodanie takich treści. Dzięki wprowadzeniu takiej funkcjonalności aplikacja, może stać się bardziej popularna, co zwiększyłoby przychody z reklam.

Tak jak zostało wspomniane w sekcji \ref{summary}, warto rozważyć zastosowanie innych technologii, które mogą poprawić wydajność aplikacji. Pierwszym etapem takich działań może być zastąpienie frameworka CSS na bardziej przyjazny dla użytkownika. W tym celu autor wykorzystałby framework Mantine, który posiada wbudowaną możliwość zmiany motywu oraz jego mały rozmiar pozwoliłby także na zwiększenie wydajności aplikacji. Dodatkowo autor zastosowałby bibliotekę Tanstack-Query (dawnej React-Query) \cite{tanstack_query}, która pozwala na zastosowanie techniki Infinite Query, która pozwoliłaby na zaimplementowanie Infinite Scroll \cite{infinite_scroll} w przypadku listy telefonów oraz producentów.

W części backendowej autor rozważyłby zmianę języka i frameworka, w którym została napisana aplikacja na język TypeScript \cite{Typescript}. Aktualnie na rynku występuje kilka frameworków, które pozwalają na zbudowanie serwera. Jednym z nich jest ExpressJs \cite{express}, który pozwala na zbudowanie szybki sposób serwera. Podobnie jak Flask, ExpressJs \cite{express} jest to framework oparty na rozszerzeniach, co pozwala na rozwijanie aplikacji w bardzo szybkim tempie.

W aplikacji dodatkowo należałoby wprowadzić ograniczenia dotyczące liczby zdjęć przechowywanych na serwerze. Aktualnie aplikacja nie posiada takich ograniczeń, co może wiązać się z dużym zużyciem pamięci, co przekształca się na wyższe koszty utrzymania aplikacji. Wprowadzenie takich ograniczeń pozwoliłoby na zmniejszenie kosztów utrzymania aplikacji.
