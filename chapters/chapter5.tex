\section{Podsumowanie}
Cel pracy został osiągnięty, aplikacja została wyposażona wszystkie założone w rozdziale \ref{ideal_solution} funkcjonalności. Po wykonaniu aplikacji zostały sformułowane problemy, które można poprawić w ramach dalszego rozwijania aplikacji. Warto je podsumować i zaplanować dalsze kroki. Kroki te zostaną opisane w rozdziale \ref{future_plans}. 

Dużo do życzenia pozostawia wygląd aplikacji. Bootstrap jest to framework, który pozwala na szybkie zbudowanie aplikacji, ale nie jest on idealny. Warto byłoby zastanowić się nad wyglądem aplikacji, aby była bardziej przyjazna dla użytkownika. W przypadku wymiany frameworka, należałoby również zastanowić się nad wyglądem aplikacji. Dużo do życzenia pozostawia całość wrażeń z aplikacji. Warto byłoby poprawić walidację pól w formularzach, aby użytkownik miał lepszą wiedzę o tym, co jest niepoprawne. Warto też, poprawić zarządzanie formularzami poprzez dodanie biblioteki do zarządzania formularzami. Przykładami takich bibliotek jest \textit{Formik} \cite{Formik} lub \textit{React Hook Form} \cite{React_hook_form}.

Autor pracy, rozważyłby także inne technologie, które mogłyby być użyte w aplikacji. Przykładem takiej technologii jest NextJS, który pozwala na szybsze budowanie aplikacji. Dodatkowo pozwala na łatwiejsze przeprowadzenie mutacji danych na stronie, bez konieczności przeładowania strony. Zaletą także tej technologii jest \textit{Server Side Rendering}, który pozwala na szybsze ładowanie strony, gdyż nie jest ona generowana po stronie klienta.

Serwer aplikacji został zbudowany na podstawie frameworka Flask, spowodowało to, że aplikacja jest bardzo łatwo modyfikowalna. Jednakże nie jest on idealny, gdyż jego główną wadą jest jego wydajność. W zastosowaniu przy tak dużej ilości danych, wydajność pozostawia dużo do życzenia. Rozważyłbym także zastosowanie serwera GraphQL, który pozwoliłby na szybsze pobieranie danych z serwera. Pozwala to na łatwe modyfikowanie danych w aplikacji, bez konieczności pobierania wszystkich danych z serwera.

\section{Wnioski}


\section{Plany na przyszłość}\label{future_plans}
