\section*{Streszczenie}
W dzisiejszych czasach na rynku urządzeń mobilnych rocznie pojawiają się setki urządzeń. Jednakże znaczna część z nich nie jest w stanie zaspokoić potrzeb użytkowników. Aby wybrać odpowiedni telefon, należy dokonać analizy swoich potrzeb tak, aby móc wybrać odpowiedni model. Nie każdy też może pozwolić sobie na kupno urządzenia nowego.

Każdego dnia na portalach aukcyjnych pojawiają się tysiące telefonów używanych. Wśród nich znajdują się również urządzenia, które nie spełniają oczekiwań użytkowników. W internecie także brakuje informacji na ich temat.

Celem niniejszej pracy było zaprojektowanie i zaimplementowanie aplikacji webowej, która umożliwiłaby użytkownikom łatwą wymianę informacji na temat używanych urządzeń mobilnych. W pracy przedstawiono konkurencyjne rozwiązania, przypadki użycia, wymagania funkcjonalne i niefunkcjonalne oraz najbardziej interesujące miejsca implementacji.

Słowa kluczowe: aplikacja webowa, Flask, Firebase, React, Telefony komórkowe

\section*{Abstract}
Nowadays, hundreds of devices appear on the mobile device market every year. However, a significant number of them are not able to meet the needs of users. To choose the right phone, you need to analyze your needs so that you can choose the right model. Also, not everyone can afford to buy a new device.

Every day, thousands of used phones appear on auction sites. Among them there are also devices that do not meet the expectations of users. The Internet also lacks information about them.  

The purpose of this thesis was to design and implement a web application that would allow users to easily exchange information about used mobile devices. The paper presents competing solutions, use cases, functional and non-functional requirements and the most interesting implementation of this application.

Keywords: web application, Flask, Firebase, React, smartphone
